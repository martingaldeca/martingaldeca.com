\begin{center}
  \textbf{\LARGE{El liberal libertario}}
\end{center}
\label{cap:Respuesta}

\justifying
   
   \section*{Preámbulo del escrito}

   
        Llevo más de cinco meses sin escribir de política, desde que he sido padre he estado bastante ajetreado y he tenido que dejar aparcado, entre otros placeres, el escribir de política regularmente.

        Aunque han surgido cientos de situaciones y cientos de temas sobre los que me hubiera gustado escribir, este en concreto me parece uno muy interesante para retomar el hábito.

        Llevo bastante tiempo pensando que, aunque me encanta debatir y especialmente con gente que no es de mi cuerda ideológica, ya que me permite expandir mis ideas y contrastar con otras formas de pensar, cada vez me es más complicado. Supongo que en parte también es por mi, y cuando identifico que así pueda ser pido disculpas y asumo mis errores cuando los localizo, además de eso, lo que veo es que se va volviendo más difícil con el tiempo encontrar espacios donde poder hablar de estos temas complejos, sin que la conversación termine de forma abrupta o con malas formas.

        Hace poco con respecto a cuando redacto este escrito, estuve en una boda, y se dio una situación muy particular. Un muy buen amigo, estando sentados en la mesa me llamó e inició una conversación con un tercero.

        \textbf{Mi amigo}: ''\textit{Martín, quiero que discutas con X}''
        
        \textbf{Yo}: ''\textit{¿Por que quieres que discuta con X?}''
        
        \textbf{Mi amigo}: ''\textit{Porque el es muy facha y tu eres muy rojo}''
        
        \textbf{X}: ''\textit{Yo no soy muy facha}''
        
        \textbf{Yo}: ''\textit{Yo no soy muy rojo}''

        En ese momento, lo primero que pensé fue "Genial, alguien que no es de mi cuerda ideológica con quien voy a poder tener un debate constructivo, con el que voy a poder aprender a ver otros puntos de vista, quizá esta conversación sea constructiva y agradable.

        \textbf{X}: ''\textit{Yo no soy muy facha, yo soy ....}''

    \section*{Liberal libertario}

        No era la primera vez que me topaba con esa situación, ya había tenido encuentros con otros liberales libertarios, y mi experiencia casi siempre había sido similar. Posiciones cerradas muy muy marcadas, sin ningún tipo de apertura al dialogo y buscando en muchos momentos la descalificación directa o indirecta desde un supuesto pedestal moral autoimpuesto.

        No obstante, como persona coherente que intento ser, siempre tiendo la mano a que puedan existir debates amistosos con todo tipo de individuos, siempre que respeten otras opiniones y que estén con predisposición a escuchar ideas con las que no comulguen a priori. He tenido debates maravillosos con sujetos liberales, extremadamente conservadoras, comunistas, anarquistas de toda cuerda (incluso paleoliberales) y hasta en una ocasión pude tener un debate con un franquista que no terminó en un desastre. Así que decidí seguir un poco la conversación.

        \textbf{Yo}: ''\textit{Ah, bueno, yo creo que soy algo bastante contrario a eso}''

        \textbf{X}: ''\textit{Ya, osea que tu en el gráfico de Nolan estás justo abajo}''

        En ese momento, como expondré un poco más adelante, ya supe perfectamente que la conversación no llegaría a ningún lado, así que simplemente esbocé una sonrisa falsa y volví a mi plato. Pero justo después pensé, bueno, vamos a dar el beneficio de la duda, quizá al yo decir que estaba en una posición contraria he predispuesto la conversación, yo y no él, a que la conversación termine.

        \textbf{Yo}: ''\textit{En realidad en términos económicos yo defiendo la \textit{Teoría Monetaria Moderna}}''

        No estoy seguro de si no me escucho, no me quiso escuchar, o simplemente lo ignoró, pero enseguida me dijo:

        \textbf{X}: ''\textit{Pero entonces a ver, tu por ejemplo el \textit{CECOT} del \textit{Salvador}, ¿bien o mal?}''

        Ingenuo de mí pensé, vale, es liberal libertario y ya ha insinuado que soy totalitario, por lo que aquí podemos coincidir y empezar a reconstruir la conversación desde una postura común.

        \textbf{Yo}: ''\textit{Claramente, mal también}''

        \textbf{X}: ''\textit{Nooooo, el CECOT es bien, hay que traerlo a España}''

        Creo que se me debió descomponer un poco la cara, así que viendo que la conversación no iba a prosperar y que estaba delante de alguien con unas convicciones algo complejas (por decirlo de forma suave), decidí ser tajante en la conversación para finalizarla.

        \textbf{Yo}: ''\textit{Ya, a mi me gusta que se respeten los derechos humanos}''

        Resultaba, que de nuevo, me encontraba con un liberal libertario con unas ideas extremadamente cerradas con quien no quería debatir.

    \section*{Los libertarios}

        Empecemos desde la primera frase que me chirrió de la conversación, y la que da título a este escrito ''\textit{Yo soy liberal libertario}''.

        Algo que siempre me ha parecido, cuanto menos, gracioso, es que muchas veces los que se consideran liberales libertarios, ni siquiera saben de donde proviene el término. Y aunque no estoy seguro, sospecho que muchos no lo abanderarían tan a la ligera y con tanto orgullo si supieran que la primera vez que se utilizó el término fue en mil novecientos setenta y dos por \textit{Michel Clouscard} usándolo como sinónimo de \textit{neofascismo} \cite{wikipedia:liberal-libertario}. 

        Si bien no comulgo en buena parte de los postulados amparados bajo el materialismo histórico de la obra de \textit{Clouscard}, considero cuanto menos acertada su visión al respecto de parte del término, y citando la propia referencia, de como habla al respecto de la confusión entre libertad y liberalización. 

        Por alguna razón, gente que por sus pensamientos se consideraría simplemente libertarios, asumen como correcto el término \textit{liberal libertario}, ya que al parecer añadirle liberal a algo hoy en día parece que te da un estatus de más intelectual, o asemeja un aire de superioridad moral. Hay que reconocer que, por desgracia, la batalla por las palabras en el vocabulario colectivo la va ganando la escuela austriaca de economía.

        Algo también peculiar de los libertarios es su bandera, ya que la \textit{Bandera de Gadsden}\cite{wikipedia:bandera-gadsden} fue creada por \textit{Christopher Gadsden}. Recordemos que esta es la bandera de la gente que se hace llamar libertarios, que etimológicamente hablando se define como ''\textit{persona que defiende la libertad absoluta}''\cite{etimologias:libertario}. Por lo que entra un poco en contradicción con el hecho de que este sujeto, fue uno de los mayores esclavistas estadounidense del siglo \textit{XVIII\cite{jornada:museo-charleston-descendientes-esclavos}}. Cuando alguna vez he mencionado este tema a algún libertario o paleoliberal su respuesta solía ser la misma ''\textit{es que en esa época lo normal era tener esclavos y no era como ahora}''. Frase que es incorrecta, ya que en aquélla época, al igual que ahora si había consciencia de lo mala, injusta y moralmente reprochable que era la esclavitud. Incluso en \textit{Estados Unidos}, donde algunas comunidades, muchas veces  cristianas protestantes, ya se manifestaban y movían en contra de la esclavitud, como por ejemplo los \textit{cuáqueros}\cite{wikipedia:sociedad-religiosa-de-los-amigos}. E incluso asumiendo como cierta la frase (que no lo es), sería como si un animalista hoy en día decidiera usar la esvástica porque \textit{Adolf Hitler} era animalista \cite{wikipedia:vegetarianismo-hitler}, y de todas formas en su época en su región se llevaba lo de ser nacional socialista.

        Otro tema importante a tratar con respecto a los liberales libertarios, al menos en \textit{España}, es que por alguna razón, suelen ser monárquicos. Desconozco si \textit{X} lo era, pero he visto a más de uno y más de dos autodenominados liberales libertarios, liberales y libertarios que son además monárquicos. Y la disonancia se expone sola. La monarquía es literalmente un estado personado en un individuo o conjunto de individuos contra su voluntad, es utilizar el máximo poder como estado para que a un ciudadano, desde su nacimiento se le imponga un camino que debe seguir, se le impongan unos protocolos y unas formas con único fin de servir al estado. Y aunque es cierto que llegado el momento pueden abdicar o rechazar el trono, llevan literalmente desde su nacimiento siendo adoctrinados y preparados para servir al estado bajo un supuesto derecho divino. No existe libertad ahí, no existe opción libre, es el estado personalmente adueñándose de la libertad de un individuo desde nacimiento. Y recordemos que los libertarios abogan por la no intervención del estado en la vida de los ciudadanos y una supuesta ''\textit{libertad}'' máxima.

        En cualquier caso, esas no suelen ser ni las únicas ni las más acuciantes contradicciones epistemológicas de los libertarios, y mucho menos no era la única para el interlocutor con el que intercambié las breves frases.

    \section*{Llamemos totalitaria a la gente, seguro que les agrada}

        Vale, en este momento ya se había empezado a torcer la conversación que acababa de empezar, y solo nos habíamos intercambiado seis frases, pero las siguientes dos frases fueron el verdadero bloqueador de entendimiento.

        \textit{X} sabía perfectamente como funcionaba el \textit{gráfico de Nolan}, apostaría a que desde que lo descubrió lo ha utilizado múltiples veces en sus discusiones para dar credibilidad y justificación a su discurso. Y si bien es de agradecer que haya gente que desestime en argumentaciones los posicionamientos unidimensionales en términos de ''\textit{izquierda}''/''\textit{progresismo}'' y ''\textit{derecha}''/''\textit{conservadurismo}'', usar el \textit{gráfico de Nolan} como argumento de peso no les deja en una posición mucho mejor.

        Se trata de la siguiente representación bidimensional rotada:

        \begin{figure}[H]
            \centering
            \includegraphics[width=0.75\linewidth]{Imagenes/Diagrama_de_nolan.png}
            \caption{Gráfico de Nolan}
            \label{fig:grafico-nolan}
        \end{figure}

        Cuando \textit{X} me dijo ''\textit{Ya, osea que tu en el gráfico de Nolan estás justo abajo}'', lo que me estaba llamando a la cara era totalitario. Quizá esperaba que no conociera esta representación política, o que le diera la satisfacción de decir que no lo conocía para que él, pudiera abiertamente explicarme lo que era y porqué era muy superior a las representaciones políticas lineales, y puesto que estábamos en una boda en mitad del banquete, mi hijo estaba justo a mi lado y no quería calentar un ambiente festivo simplemente sonreí y volví a mi plato.

        Esta representación está mal desde la base, aunque cuando investigas un poco sus orígenes es más fácil perfilar el motivo.

        El \textit{gráfico de Nolan} fue creado por \textit{David Nolan} en el año mil novecientos sesenta y nueve\cite{wikipedia:grafico-nolan}. \textit{David Nolan}, sí, quien es el fundador del \textit{Partido Libertario} de los \textit{Estados Unidos}\cite{wikipedia:david-nolan}. Podemos decir que es una persona, ciertamente sospechosa de querer dejar a los libertarios en una posición moral superior.

        Y es que, si miramos bien el gráfico podemos ver visualmente el posicionamiento desde el principio. Vamos por ejemplo a hablar de la representación bidimensional de la \textit{función Gaussiana}\cite{wikipedia:funcion-gaussiana}, es decir, la que obviamente se representa de la siguiente forma:

        \begin{figure}[H]
            \centering
            \includegraphics[width=0.75\linewidth]{Imagenes/gaussiana_rotada.png}
            \caption{Función Gaussiana rotada}
            \label{fig:funcion-gaussiana-rotada}
        \end{figure}

        Como se puede ver, el girar los ejes (que ya hablaremos más adelante sobre ellos), es una cosa bien premeditada, para que coincida exactamente la posición que el autor quería defender como la cúspide de las ideologías, dejando en la parte contrapuesta, o, en palabras de \textit{X}, en la parte que está justo abajo, el espectro político totalitario.

        No se, como punto de partida para una conversación con alguien que no conoces no me parece lo más acertado llamarle totalitario a la cara y esperar que la conversación siga tranquilamente.

        En cualquier caso, está gráfica, manoseada para dejar en una posición claramente superior al liberalismo y al libertarismo, no deja de ser una representación extremadamente pobre del espectro político. La elección de los ejes es cuanto menos cuestionable, ya que lo único que pretenden representar es el posicionamiento con respecto a la ''\textit{libertad}'' en dos diferentes dimensiones. Y he aquí uno de los grandes problemas que encuentro con los libertarios, y es su denostada idea de la libertad. 

        Algunas de las variantes de la \textit{gráfica de Nolan} son si cabe más descaradas en su posicionamiento ideológico base, esta es una de mis favoritas.

        \begin{figure}[H]
            \centering
            \includegraphics[width=0.75\linewidth]{Imagenes/Political-spectrum-multiaxis.png}
            \caption{Political Spectrum Multiaxis}
            \label{fig:bastardos}
        \end{figure}

        Las facciones izquierdistas, acorde a los que defienden este gráfico, no se centra ni económicamente en la comunidad ni culturalmente en el individuo, al contrario que las facciones de la derecha, que se centran plenamente en ambos ejes. En fin, un ejemplo de manual de falacia maximalista.

        Y como mencionaba en el capítulo anterior, ya \textit{Michel Clouscard} hablaba del problema del término de la libertad, y desde luego no fue el primero ni el último, pues \textit{Platón} ya nos habló en sus \textit{Diálogos} sobre como su supuesto maestro \textit{Sócrates} hablaba antaño en profundidad de lo que podía significar la libertad\cite{nueva-acropolis:opinaba-socrates-libertad}.

        Y es que para los libertarios la libertad está relacionada con que un tercero o grupo de terceros (generalmente representados como un estado) no interfieran en tus asuntos siempre y cuando tu no interfieras en los de los demás. Resumida con la frase ''\textit{mi libertad termina donde empieza la tuya}''. Con esta premisa van en contra de la recaudación de impuestos, de la redistribución de la riqueza, de la sanidad y la educación pública, del transporte público, en general de todo lo que tenga detrás la palabra público o redistributivo.

        Asumen que quien lo defendemos es porque nos gustan ''\textit{las paguitas}'' con las que llaman de forma despectiva a los subsidios públicos para los que más lo necesitan. Y en una gran cantidad de casos se da una de estas dos opciones, o tienen suficiente dinero como para las transferencias de capital les sean más costosas que las prestaciones directas o indirectas que reciben. O bien, porque aunque reciban mucho más de lo que entregan en impuestos, han sido manipulados y engañados para hacerles pensar que les roban.

        La mayor parte, al menos acorde a mi experiencia se encuentran en el segundo grupo. Podemos ver como mirando los datos que la inmensa parte de la población recibe mucho más en prestaciones directas que lo que entrega en impuestos\cite{fedea:observatorio-reparto-impuestos-prestaciones}. Estos datos sin ir más lejos son proporcionados por la \textit{Think Tank} \textit{Fedea}, presidido por \textit{José Ignacio Goirigolzarri}, el expresidente de \textit{Bankia}, expresidente del \textit{Banco Financiero y de Ahorros} y ex-consejero delegado del \textit{BBVA} \cite{wikipedia:jose-ignacio-goirigolzarri} y dirigido por \textit{Ángel de la Fuente} \cite{wikipedia:angel-de-la-fuente}, que llego a esa posición nombrado a dedo por el gobierno de \textit{Mariano Rajoy} para evitar informes desfavorables al gobierno del \textit{Partido Popular}\cite{eldiario:director-fedea-criticar-gobierno}

        \begin{figure}[H]
            \centering
            \includegraphics[width=1\linewidth]{Imagenes/DatosPrestaciones.png}
            \caption{Distribución conjunta de impuestos y prestaciones en los hogares españoles en 2021 (tipos y subsidios medios en porcentaje de la renta bruta)}
        \end{figure}

        En estos datos, podemos mirar como el ochenta por ciento de las personas aproximadamente salen ganando con el estado. Y eso solo en ayudas directas, porque estoy seguro de que aunque el veinte por ciento restante no disfrute de tantas ayudas directas, los beneficios que trae a la sociedad esas ayudas superan con creces en la mayoría de casos los impuestos que deben pagar.

        Aunque podría ponerme a bucear en profundidad sobre \textit{Teoría Monetaria Moderna}, y de como hasta incluso algunos liberales asumen el \textit{Chartalismo} como válido, creo que podría escaparse del alcance de este escrito.

        En cualquier caso, una manera relativamente sencilla de exponer las confrontaciones directas con la realidad que postulan los liberales y en particular los libertarios, es enfrentarlos a los propios datos sobre los países con los estados más grandes y más pequeños del mundo, entendiendo el tamaño bajo los términos de porcentaje del PIB empleado en gasto público, que aunque no sea una medida que crea del todo válida bajo mi prisma ideológico, si parece chocar en extremo con las creencias de los liberales y los libertarios.

        En la cabeza de países con un mayor tamaño del estado, excluyendo países en guerra como \textit{Ucrania} (por estar artificialmente alto y micro-naciones como \textit{Nauru} o \textit{Kiribati} nos encontramos con \textit{Italia}, \textit{Francia}, \textit{Grecia}, \textit{Austria}, \textit{Finlandia}, \textit{Eslovenia}, \textit{Luxemburgo}, \textit{Hungría} y \textit{Reino Unido}. En contraposición, los países con un tamaño del estado más pequeño son \textit{República Federal de Somalia}, \textit{Etiopía}, \textit{Líbano}, \textit{Bangladesh}, \textit{Sudán}, \textit{Tayikistán}, \textit{Guinea Ecuatorial}, \textit{Madagascar}, \textit{República Centroafricana} y \textit{Malí}\cite{worldbank:gc.xpn.totl.gd.zs-2024}.

        No les veo nunca solicitando nacionalidad es esos países y deseando mudarse para tener un estado más pequeño. Y en las ocasiones en las que han intentado poner en práctica sus ideas, han terminado frustradas por los osos como ocurrió en \textit{Grafton}\cite{bbc-mundo:utopia-libertaria}. ¿Qué podría salir mal cuando eliminas todos los servicios públicos de una ciudad?, bueno, ahora sabemos que algo que podría pasar es que te invadan los osos. 

    \section*{A problemas complejos, soluciones simples}

        Esto es algo que ocurre con una frecuencia notable, no solo en el tema del espectro político, sino en general. Lo que más encuentro es a gente que sucumbe ante el efecto \textit{Dunning-Kruger}\cite{wikipedia:efecto-dunning-kruger}. Y siendo absolutamente sinceros, yo mismo estuve inmerso a ese efecto cuando empecé a interesarme en la política. Pasé por posiciones que podrían considerarse extremadamente conservadoras, posiciones liberales, posiciones anarquistas, incluso cuando caí en la desesperanza de entender que mi conocimiento era absolutamente escaso, pasé por posiciones marxistas. Y sobre todo al principio, estaba completamente seguro de que tenía la razón, de que no podía estar equivocado porque era evidentemente correcto mi posicionamiento, era imposible contra argumentarlo, porque era perfecto, debía serlo y las posiciones de los demás se basaban en manipulaciones y mentiras.

        \begin{figure}[H]
            \centering
            \includegraphics[width=0.5\linewidth]{Imagenes/efecto-Dunning-Kruger.jpg}
            \caption{Efecto Dunning-Krugger}
            \label{fig:dunning-kruger}
        \end{figure}

        Con esto, espero que no se me mal interprete, sigo considerando que mi conocimiento es ínfimo en la inmensa mayor parte de áreas del conocimiento sino en todas. Y ahora considero que mis posicionamientos se basan muchas veces en unos cimientos de barro por esa falta de conocimiento, por lo que siempre estoy abierto a aprender cosas nuevas y escuchar otras ideas que puedan contrastar con las mías, y si el argumentario utilizado es convincente, no solo de escucharlas sino de integrarlas.

        Cuanto más he ido aprendiendo, más me he dado cuenta de que quienes intentan darte soluciones muy simples a problemas muy complejos, es porque quieren engañarte, o porque han sido engañados en su ignorancia, y debido al efecto \textit{Dunning-Krugger} perpetúan el engaño con otros.

        El espectro político, por supuesto que no debería representarse en términos de izquierda o derecha, y tampoco no debería representarse simplemente con una gráfica con dos ejes, y encima que ambos ejes hablen solo de la ''\textit{libertad}''. Los seres humanos somos infinitamente más complejos con respecto a las posiciones morales y políticas que mantenemos, aunque cada vez tendamos a pensar menos sobre ellas y a aceptar las simplificaciones sin pestañear.

        Si bien algunas representaciones posteriores al \textit{gráfico de Nolan} corrigieron parte de las falsas dicotomías producidas por el mismo, y diferenciaron mejor los ejes para que no estuvieran íntimamente relacionados, se suiguen quedando alejadas de una forma adecuada de representar la realidad. Un par de ejemplos a esto son la \textit{Brújula política} propuesta por la \textit{Political Compass Organization}\cite{politicalcompass:crowdchart2}, o el diagrama creado por \textit{Tomás de la Fuente}, conocido como ''\textit{elgentilhombre}'' que supera en complejidad a la \textit{Brújula política}\cite{elgentilhombre:test}. En este último, los ejes están elegidos de forma que el paleoliberalismo quede arriba alejado del origen, y en general esto se repite para el espectro de las conocidas como ''\textit{derecha}'', quedando encima de las izquierdas y sobretodo, del comunismo postmarxista. Y tampoco es de extrañar, ya que \textit{Tomás de la Fuente} es un paleoliberal declarado.

        \begin{figure}[H]
            \centering
            \begin{minipage}{0.40\textwidth}
                \centering
                \includegraphics[width=\linewidth]{Imagenes/brujula_politica.png}
                \caption{Brújula política}
                \label{fig:brujula-politica}
            \end{minipage}
            \begin{minipage}{0.40\textwidth}
                \centering
                \includegraphics[width=\linewidth]{Imagenes/diagrama_zeto.png}
                \caption{Diagrama Zeto}
                \label{fig:zeto}
            \end{minipage}\hfill
        \end{figure}

        La realidad es que medir la complejidad del espectro político es harto complicado, sino rozando los limites de lo imposible, incluso haciendo algunas abstracciones. Y es que en última instancia el espectro político debe representar el ideario de un individuo, y este depende de una ingente cantidad de factores, que, para más inri pueden variar a lo largo del tiempo. Y no solo eso, sino que esos factores en muchos casos son tan subjetivos como la idea de libertad.

        Una visualización del espectro político que yo podría pensar que es más acertada, tiene más que ver con un análisis multidimensional de distintos factores socio-económico-culturales, que además variaría con el tiempo y que por cada uno de las dimensiones existe un margen de aceptabilidad entorno al valor ideal de cada individuo. Ya que los humanos, generalmente, no somos extremadamente rígidos, y podemos aceptar y estar conformes para llegar a acuerdos con posiciones ideológicas cercanas aunque no sean exactamente las nuestras. 

        Una representación política algo más acertada, por tanto, tiene este aspecto, aunque esta sea una lista extremadamente corta de posibles dimensiones que estimo que podrían ser cientos sino miles.

        \begin{figure}[H]
            \centering
            \includegraphics[width=0.75\linewidth]{Imagenes/controles_ideológicos.png}
            \caption{Representación de ideología incompleta}
            \label{fig:ideologia}
        \end{figure}

        Solo para que quede claro, esos valores no corresponden a mis opiniones, sino que son el resultado proporcionado por la IA utilizada para generar la imagen\cite{google:stitch}.

        En cualquier caso, volvamos a hablar de la conversación con \textit{X}.

    \section*{Quienes aman la ''libertad'', a costa de la vida de otros}

        Como comenté al principio, en el momento en el que llegó la pregunta sobre el \textit{CECOT}, pensé que la conversación podría reanudarse desde una posición que claramente compartiríamos. Porque, ¿cómo no íbamos a compartir la postura sobre el \textit{CECOT}? \textit{X} se había declarado abiertamente liberal libertario, una persona que pondría la ''\textit{libertad}'' por encima de todo lo demás, que aborrecería la figura del estado, y en particular de un estado totalitario. Al fin y al cabo, el mismo había utilizado el \textit{gráfico de Nolan} para intentar hacerme ver que el totalitarismo era una posición alejada de la suya. 

        Así que respondí con sinceridad y con una sonrisa, pensando lo anterior.

        \textbf{Yo}: ''\textit{Claramente, mal también}''

        \textbf{X}: ''\textit{Nooooo, el CECOT es bien, hay que traerlo a España}''

        Citando a \textit{Plauto}, ''\textit{En una mano lleva la piedra, y con la otra muestra el pan}''. Y es que es algo que encuentro frecuente en perfiles libertarios y perfiles neoreaccionarios, a la cara te sacan a relucir valores de libertad, y en cuanto tiras del hilo que sujeta sus caretas encuentras opresión ejercida con máximo poder desde un estado totalitario.

        Se les llena la boca diciendo que van a achicar el estado, cerrar ''\textit{chiringuitos}'' y subvenciones, y en cuanto tocan el poder reducen en las partidas presupuestarias de bienestar social mientras aumentan de forma desproporcionadas las relacionadas con la represión. El propio \textit{Nayib Bukele} era al principio de su primera legislatura la ''\textit{rock-star}'' de los liberales y de los libertarios. Aquél que buscaba la ''\textit{libertad}'' de sus ciudadanos convirtiéndose en el primer país del mundo en convertir el \textit{Bitcoin} en moneda de curso legal\cite{bbc-mundo:el-salvador-bitcoin}, todos en distintos países del mundo hablaban de él.

        Ahora que tiene afianzado el poder, y que ha podido modificar la constitución para poder ser reelegido de forma indefinida y añadirse un año cada mandato sin elecciones\cite{abc:bukele-reeleccion}, que ha podido quitarse de encima a más de ciento cincuenta jueces en el país a los que ha sustituido con leales\cite{gaceta:bukele-poder-judicial}, y que, para decepción de muchos liberales y libertarios que le tenían en un pedestal, ha retirado el \textit{Bitcoin} como moneda de curso legal solo cuatro años después de aprobarla\cite{bbc-mundo:c4gpv776zd0o}. 
       
        Ahora, algunos, quizá los más honestos con su posición liberal o libertaria, empiezan a renegar de él. Pero la realidad es que la mayoría simplemente muestran sus verdaderas ideas totalitarias y entran en contradicciones epistemológicas extremas, como hizo \textit{X} en el momento que defendió el \textit{CECOT} de \textit{El Salvador}.

        El \textit{CECOT} es el denominado \textit{Centro de Confinamiento del Terrorismo}\cite{wikipedia:cecot}. Un lugar en el que hacinan a parte de la población salvadoreña desde hace años, sin posibilidad alguna a reinserción, y donde los derechos humanos son opcionales. 

        Cuando miramos los datos, efectivamente la presencia de las \textit{Maras} en \textit{El Salvador} se ha reducido considerablemente hasta casi su extinción, y efectivcamente la tasa de homicidios se ha desplomado a mínimos históricos. ¡Incluso \textit{Estados Unidos} bajo la \textit{Administración Trump} acaba de proclamar que es más seguro viajar a \textit{El Salvador} que viajar a \textit{España}!\cite{vozpopuli:el-salvador-seguro-eeuu}

        \textit{El Salvador} en los años previos a \textit{Bukele} era un infierno, eso es innegable, no obstante, podemos mirar los datos, a ver que nos dicen sobre el relato creado en torno a la seguridad del país.

        Un gráfico que me llama poderosamente la atención es este\cite{wikipedia:homicidios-el-salvador}:

        \begin{figure}[H]
            \centering
            \includegraphics[width=1\linewidth]{Imagenes/homicidios-salvador.png}
            \caption{Homicidios en el Salvador 2000-2024}
            \label{fig:homicidios}
        \end{figure}
       
        El presidente \textit{Bukele} redujo en su primer mandato entre dos mil diecinueve y dos mil veinticuatro un noventa y cinco por ciento los homicidios, pasando de dos mil trescientas noventa y ocho en dos mil diecinueve a ciento catorce en dos mil veinticuatro. Y no quiero que se malinterprete, esta reducción es algo bueno, y además los cuatro años anteriores, bajo el mandato de \textit{Salvador Sánchez Cerénel} pasaron de seis mil seiscientos cincuenta y seis a dos mil trescientos noventa y ocho, que aunque es una caída del sesenta y tres por ciento, son casi dos mil homicidios evitados más que los que consiguió \textit{Bukele} en el mismo tiempo. Y que conste que con esto no estoy ensalzando la figura de \textit{Salvador Sánchez Cerénel} sobre el cual pesan graves acusaciones de corrupción, sino constatando que \textit{Bukele} no hizo más que continuar una tendencia.

        Pero no todos los delitos son homicidios, ¿qué ocurre cuando exploramos los datos de otros delitos registrados en \textit{El Salvador} para las mismas fechas? Bueno, lo primero que encontramos es una cosa muy curiosa, y es que a partir del segundo mandato del presidente dejan de publicarse datos al respecto, ya que se obtenían del \textit{Ministerio de Seguridad Pública y Justicia}\cite{seguridad-sv:ministerio} del gobierno de \textit{El Salvador} y de \textit{Dirección de Información y Análisis (DIA)}\cite{seguridad-sv:dia} que operaba bajo el mismo ministerio.

        Para el segundo caso, los últimos datos publicados son directamente del dos mil veintidós, incluso cuando su misión declarada en cabecera es ''\textit{Desarrollarnos como la entidad responsable de consolidar, analizar y sistematizar información en materia de seguridad de manera oportuna, eficiente y eficaz.}'' Y para el \textit{Ministerio de Seguridad Pública y Justicia} encontramos que cuando intentas acceder a la información pública a través del apartado ''\textit{contáctenos}'', la respuesta de los navegadores es la siguiente:

        \begin{figure}[H]
            \centering
            \includegraphics[width=0.75\linewidth]{Imagenes/navegador_no_funciona.png}
            \caption{Respuesta del navegador a las solicitudes de información}
            \label{fig:chungo}
        \end{figure}

        En cualquier caso, con los datos de su primer mandato recopilados por la \textit{Organización de los Estados Americanos}\cite{oas:ios-el-salvador}, encontramos lo siguiente:

        \begin{figure}[H]
            \centering
            \begin{minipage}{0.32\textwidth}
                \centering
                \includegraphics[width=\linewidth]{Imagenes/delitos_violencia_sexual.png}
                \caption{Delitos lesivos de naturaleza sexual por cada 100.000 habitantes}
                \label{fig:delitos-sexuales}
            \end{minipage}\hfill
            \begin{minipage}{0.32\textwidth}
                \centering
                \includegraphics[width=\linewidth]{Imagenes/delitos_agresion_sexual.png}
                \caption{Delitos de agresión sexual}
                \label{fig:delitos-agresion-sexual}
            \end{minipage}\hfill
            \begin{minipage}{0.32\textwidth}
                \centering
                \includegraphics[width=\linewidth]{Imagenes/delitos_agresion_grave.png}
                \caption{Delitos agresión grave}
                \label{fig:delitos-agresion-grave}
            \end{minipage}            
        \end{figure}

        \begin{figure}[H]
            \centering
            \begin{minipage}{0.32\textwidth}
                \centering
                \includegraphics[width=\linewidth]{Imagenes/delitos_corrupción.png}
                \caption{Delitos por corrupción}
                \label{fig:delitos-corrupcion}
            \end{minipage}\hfill
            \begin{minipage}{0.32\textwidth}
                \centering
                \includegraphics[width=\linewidth]{Imagenes/delitos_hurto_por_cien_mil.png}
                \caption{Delitos por hurto por cada 100.000 habitantes}
                \label{fig:delitos-hurto}
            \end{minipage}\hfill
            \begin{minipage}{0.32\textwidth}
                \centering
                \includegraphics[width=\linewidth]{Imagenes/delitos_trafico_de_personas.png}
                \caption{Delitos por tráfico de migrantes}
                \label{fig:delitos-trafico-migrantes}
            \end{minipage}
        \end{figure}

        Eso sí, los homicidios en extranjeros se han reducido hasta prácticamente ser nulos
        
        \begin{figure}[H]
            \centering
            \includegraphics[width=0.75\linewidth]{Imagenes/extranjeros_homicidios.png}
            \caption{Ciudadanos extranjeros, víctimas de homicidio intencional}
            \label{fig:homicidios-a-extranjeros}
        \end{figure}

        Así que podemos agradecer a la \textit{Administración Trump} estadounidense por reducir el riesgo de viajar a \textit{El Salvador}, catalogándolo como país más seguro que por ejemplo \textit{España}. Quizá, y solo quizá, también esté relacionada esa decisión con el hecho de que sabemos que la \textit{Administración Trump} pagó casi cinco millones a \textit{El Salvador} para que recibiera a deportados estadounidenses y se encargara de, en muchas ocasiones, confinarlos en el \textit{CECOT}\cite{elpais:trump-pago-deportados}.

        Cierto es que ahora las calles en \textit{El Salvador} tienes muchas menos posibilidades de ser asesinado que antes, si eres un ciudadano promedio no delincuente, o si no eres un opositor o un periodista crítico con el régimen totalitario, como por ejemplo \textit{Víctor Barahona}\cite{ifj:apes-barahona} o como \textit{Ingrid Escobar} que tuvo que exiliarse del país\cite{elsalvador:directora-socorro-juridico-salio}. También deberías preocuparte si formas parte de una organización humanitaria, ya que es posible que termines en el \textit{CECOT} también por una detención arbitraria\cite{swissinfo:onu-rechaza-persecucion-sv}. Y bueno, si una persona termina en la cárcel, ya sea de forma arbitraria o de forma legítima, solo puedo desearles suerte. Ya que además de ser más seguro, \textit{El Salvador} se ha convertido en el país con la tasa de más alta del mundo en muertes en cárceles, siendo además el noventa y cuatro por ciento de esas personas fallecidas personas sin derecho a segunda audiencia que murieron bajo tutela\cite{efe:fallecidos-carceles-sv}.

        Y el producto interior bruto del país está creciendo con \textit{Bukele}, es cierto, a un ritmo de casi el cuatro por ciento en dos mil veinticuatro. Esto es algo que cualquier liberal aplaudiría con euforia, y muchos, sospecho que hasta pasarían por alto el que se está haciendo a costa de emplear a la población carcelaria como mano de obra esclava (recordemos que tiene el porcentaje de población carcelaria más alto del mundo). Y no es que sea algo que ocultan precisamente, el propio presidente lo difunde en sus redes hablando de ''\textit{trabajos comunitarios}''\cite{x:bukele-1885522856567537700}.        
        
        Y es que, ya lo decía \textit{Benjamin Franklin}, ''\textit{Aquellos que pueden renunciar a la libertad esencial para obtener un poco de seguridad temporal no merecen ni libertad ni seguridad}''. 

    \section*{Conclusiones}

        Me encantaría poder tener conversaciones respetuosas y enriquecedoras con personas que se declaran ''\textit{liberales libertarias}'' y así lo he intentado varias veces, aunque siempre suele llegar el punto en el que esa persona dice alguna declaración propia de la barbarie y en contra de los derechos humanos, y si bien estoy dispuesto a escuchar todo tipo de opiniones y aprender de ellas, no quiero darle voz a personas que desean traspasar esa linea.

        Dicho todo eso quiero recordar la última parte que \textit{X} dijo en nuestra conversación y resaltar las últimas palabras: ''\textit{el CECOT es bien, hay que traerlo a España}''. Hay que tener especial cuidado con que estas personas nunca accedan al poder, ya que cuando llega la hora de la verdad, aparece lo que se esconde detrás de su concepto de ''\textit{libertad}'' para quienes no piensan como ellos: represión, autoritarismo, segregación, censura y en resumen como dijo \textit{Clouscard}, \textbf{neofascismo}.

         Como siempre, agradecer que hayas leído mis reflexiones y conclusiones, y agradecer a todas las personas con las que he podido debatir y charlar sobre estos temas. Para finalizar cómo con el resto de mis escritos,
    
        Y tú, ¿qué piensas?
        
        \includegraphics{Imagenes/Firma.png}
                