\center{\textbf{\LARGE{La religión y la política }}}\label{cap:Inicio}

\justifying

    Desde hace ya un tiempo, uno de los temas recurrentes en mis conversaciones sobre política derivan hacia temas teológicos. Esto se debe en parte, y soy consciente de ello, por la gente con la que suelo discutir sobre política suele estar también interesada en teología. Sin ir más lejos uno de mis mejores amigos, al que tengo una gran estima y con quien disfruto profundamente de conversar, es estudiante de teología, por lo que es normal que ciertas conversaciones derivan hacia ese tema.

    En este escrito quiero explorar la relación, bajo mi punto de vista directa, que tiene la religión y la política. En concreto, me centraré en el cristianismo, y más concretamente en el catolicismo. Además, aunque trataré de realizar ciertas generalizaciones, no estará centrado en un sentido amplio del catolicismo sino en el catolicismo dentro de Europa y de una forma más profunda en el catolicismo dentro de España. 

    Antes de nada me gustaría dejar clara mi opinión personal para que quien lea este escrito pueda posicionar mi punto de vista. Yo no soy católico, ni siquiera soy cristiano. Si tuviera que definirme sería como agnóstico. Soy científico de estudios y mi mente estructurada de tal forma no me permite fácilmente creer en algo sin tener pruebas de su viabilidad científica con una certeza estadística o lógica muy considerable. No obstante, conozco testimonios de primera mano de gente muy poco sospechosa de mentir, que ha tenido experiencias religiosas, dentro de distintas religiones, que son para esas personas tan palpables y físicas como el teclado del ordenador con el que escribo estas palabras. Desconozco si se trataron de experiencias inducidas por sus mentes en momentos concretos, mi mente científica, y a veces cínica, me dice que es lo que más probabilidades tiene, pero no puedo cerciorarme de ello. Es por ello, que me mantengo en una postura de \textit{epojé} o suspensión de juicio, o dicho de otra forma, en la que mi creencia es que podría existir cualquier religión y o Dios que fuera real, o que podría no hacerlo. Esta creencia es la que mejor encaja con mi punto de vista.

    Por otro lado, he de mencionar que a mi se me ha educado siguiendo unos valores cristianos, en concreto católicos y que algunos de mis referentes, como por ejemplo mi padre, una de las mejores personas que conozco, sino la mejor, son devotos católicos de convicción. Por lo que en última instancia, personalmente no me parece que el catolicismo de como resultado malas personas, sino que en mi opinión, todo lo contrario.

    En el escrito voy a hablar sobre catolicismo, y generalizando muchas veces, sobre religión en sentido amplio, suele estar asociado al conservadurismo político. Si bien he dicho generalizando, es obvio que dentro de la propia Iglesia hay sectores conservadores y sectores progresistas, que pueden ir desde por ejemplo sectas ultra conservadores como \textit{El Yunke} \cite{yunke} a revistas que defienden los colectivos LGBTQ desde el seno del la Iglesia como Outreach \cite{outreach}. 
    
    Dejadme lanzar una frase sobre la que reflexionar a posteriori \textit{Si preguntas por la calle, una gran parte de la población asociará la religión con lo que se entiende en mi país por derecha.}

    Esta es una frase que a gente le habrá causado una afirmación, a otra una negación y a otra escepticismo. Por lo tanto, podemos tratar de ver los datos de los que disponemos para este tema antes de avanzar sobre reflexiones y conclusiones al respecto de esta frase. En este caso podemos utilizar los datos del estudio nº 3427 del barómetro de noviembre de 2023 del CIS \cite{barometro}. Vamos a visualizar los datos de algunas de las preguntas realizadas a la población para poder realizar mejor el análisis.

    \begin{figure}[H]
        \centering
        \includegraphics[width=\textwidth]{Imagenes/Distribucion_de_creencias_religiosas_segun_escala_de_auto-ubicacion_ideologica.png}
        \caption{Pregunta C4}
        \label{fig:PC4}
    \end{figure}
    
    \begin{figure}[H]
        \centering
        \includegraphics[width=\textwidth]{Imagenes/Distribucion_de_creencias_religiosas_segun_escala_de_auto-ubicacion_ideologica_normalizado.png}
        \caption{Pregunta 15}
        \label{fig:P15}
    \end{figure}
    
    \begin{figure}[H]
        \centering
        \includegraphics[width=\textwidth]{Imagenes/Religiosidad_por_edad.png}
        \caption{Pregunta 0B}
        \label{fig:P0B}
    \end{figure}
    
    \begin{figure}[H]
        \centering
        \includegraphics[width=\textwidth]{Imagenes/Nivel_de_estudios_por_creencias_normalizado.png}
        \caption{Pregunta C3aa}
        \label{fig:PC3aa}
    \end{figure}
    
    \begin{figure}[H]
        \centering
        \includegraphics[width=\textwidth]{Imagenes/Votacion_directa_al_parlamento_espanol.png}
        \caption{Pregunta 12R}
        \label{fig:P12R}
    \end{figure}
    
    \begin{figure}[H]
        \centering
        \includegraphics[width=\textwidth]{Imagenes/Votacion_secundaria_al_parlamento_espanol.png}
        \caption{Pregunta 13R}
        \label{fig:P13R}
    \end{figure}
    
    Lo primero de todo, si bien se podría argumentar que el \textit{CIS}, al estar bajo el mando de gobierno de turno no es una fuente fiable de datos y adulterará los mismos, se puede contra argumentar que esto solo ocurriría para poder mostrar unas votaciones más favorables hacia la izquierda que a la derecha, y en este caso, los datos han sido normalizados para que no se beneficie con ellos a ningún partido en caso de existir esa supuesta injerencia sobre los resultados de los mismos.

    Ahora podemos analizar algunos de los datos que tenemos. Vamos por tanto antes de extraer conclusiones simplemente a leer estos datos de forma objetiva.
    
    Por un lado, con un simple vistazo a la primera gráfica [\ref{fig:PC4}], podemos ver que en nuestro país hay una mayoría de católicos si juntamos a los practicantes y los no practicantes. Además, podemos ver que cuanto más nos desplazamos hacia la derecha en el espectro ideológico, más copan estadísticamente las posiciones ideológicas, siendo a partir del seis en la escala del uno al diez más del cincuenta por ciento y a partir del ocho, más del setenta y cinco por ciento. Podemos ver, por contrario, que por debajo del cinco nunca llegan a superar el cincuenta por ciento y en el caso del valor dos en la auto-ubicación ideológica, ni siquiera el veinticinco por ciento.

    En la siguiente gráfica [\ref{fig:P15}] podemos ver los datos normalizados por auto-ubicación ideológica, esto es, asumiendo que cada posicionamiento tuviera la misma representación. Aquí ocurren dos cosas, la primera es que vemos que en el grupo más escorado a la derecha se reduce el número de católicos, y en los grupos más escorados a la izquierda los católicos ni siquiera alcanzan un veinticinco por ciento, apareciendo un mayor número de agnósticos y ateos.

    Para la gráfica referida a la religiosidad por edad [\ref{fig:P0B}], podemos observar dos fenómenos, el primero es que a mayor edad el porcentaje de católicos asciende significativamente y se reduce el porcentaje de creyentes en otras religiones así como de ateos. Además algo que también aumenta significativamente con la edad es el número de personas que se abstienen de responder a la pregunta. Podemos ver que el número de agnósticos se mantiene más o menos constante con pequeñas variaciones a lo largo de los grupos.

    La gráfica que nos ilustra el nivel de estudios por creencias normalizado [\ref{fig:PC3aa}] nos muestra que a partir de secundaria de primera etapa el número de católicos se mantiene aproximadamente constante en torno al veinticinco por ciento. Podemos ver que los encuestados que respondieron que solo tienen primaria es donde aparece el mayor número de católicos, llegando este porcentaje al cincuenta por ciento. Además, algo que se puede apreciar también es que el número de agnósticos es superior en aquellos encuestados que tienen al menos secundaria de segunda etapa. Y que se reduce el número de creyentes de otra religión entre los que tienen estudios superiores.

    Con respecto a la pregunta del \textit{CIS} sobre la votación del encuestado en las elecciones al parlamento español de julio de 2023 [\ref{fig:P12R}] podemos ver que el mayor porcentaje de católicos se concentra en los votantes del \textit{PP} seguido de los de \textit{VOX}, mientras que el grupo con menos católicos es el de \textit{Sumar} junto con otros partidos. Podemos ver que el número de ateos y agnósticos es muy superior también en los grupos de \textit{PSOE} y especialmente en el de \textit{Sumar}. Algo a destacar también es que el grupo de \textit{VOX} es el que tiene más representación de creyentes de otra religión junto con los votantes en nulo y los no votantes.

    Finalmente en la ultima de las gráficas traídas a este texto sobre a quien votaría el encuestado en caso de no votar por otro partido [\ref{fig:P13R}], podemos ver que ahora la representación de los católicos se aglutina entorno a \textit{VOX}, seguido de cerca por las opciones de \textit{PP} o por no votar o votar nulo. Por otro lado los ateos, en este caso, tienen mayor representación en el voto nulo, y los que si votarían se siguen concentrando en \textit{Sumar} y más crecientemente en este caso en \textit{PSOE}. Podemos ver como el grupo de los agnósticos donde tiene menor representación es en los partidos \textit{VOX} y \textit{PP} y tienen más representación en este caso en los grupos de \textit{PSOE} y especialmente \textit{Sumar}.

    Ahora podemos intentar traer algunas reflexiones al respecto de estos datos, y posteriormente, tratar de extraer algunas conclusiones.

    La primera reflexión que cabría hacer, es por qué los católicos se acercan a la derecha de forma tan evidente en los datos. Podemos ver, especialmente en la gráfica referente a la pregunta quince [\ref{fig:P15}] y en menor medida en la pregunta doce R [\ref{fig:P12R}] que esto es algo objetivo, los datos así lo muestran. Estadísticamente, los católicos votan a la derecha. Podemos decir por tanto que la supuesta afirmación que comentamos al principio del escrito ''\textit{Si preguntas por la calle, una gran parte de la población asociará la religión con lo que se entiende en mi país por derecha.}'', es cierta y fundamentada por datos.

    Pero, ¿por qué?, ¿por qué el conservadurismo político se asocia tan bien con las ideas católicas? Bueno, aquí yo propongo mi reflexión al respecto. La pregunta está formulada en ese sentido a propósito, ya que desde un punto de vista filosófico-teológico e histórico, las ideas radicales (de raíz) católicas no se asocian tan bien al conservadurismo. El catolicismo se basa en la figura de Jesús de Nazaret, su vida y muerte, y me atrevería a decir que coincidiría con cualquier católico de convicción, ya sea conservador o progresista, que la vida de Jesús no fue conservadora, sino progresista y rupturista con respecto a su contexto socio-histórico en una grandísima cantidad de aspectos. Esto podría dar cabida a un debate teológico más abierto al respecto que estaré encantado de tener si alguna persona que lea este escrito difiere, pero para este escrito en concreto, asumiremos esto como axioma.

    Entonces, si los valores radicales del catolicismo son progresistas, ¿cómo se han tornado los católicos en conservadores?. La respuesta a mi parecer está en la propia historia. El cristianismo fue absolutamente progresista durante sus primeros trescientos años aproximadamente. Y es ahí donde rompió con el progresismo, cuando Constantino I se convierte en el primer emperador de Roma cristiano. En ese punto el progresismo cristiano llegó a una gran cantidad de gente, pero no llegó como fue concebido, sino que llegó como un medio de control estatal principalmente. Ahora se trataba de una herramienta para controlar a las masas, los gobernantes desde entonces se dieron cuenta de lo útil que resultaba. El cristianismo se fue perneando en todas las estructuras de poder europeas, y fue afianzándose entre los que ostentaban el poder y querían mantener su estatus. Lo que entendemos por edad medieval \cite{edadMedia} y antiguo régimen \cite{antiguoRegimen} estuvieron absolutamente impregnados del cristianismo, y distintas vertientes del mismo, en Europa.

    Progresistas y conservadores de aquellas épocas no asociaban directamente y de forma generalizada la religión a ninguno de los conceptos de progreso o conservadurismo, de la misma forma que ahora no asociaríamos el concepto de internet directamente de forma generalizada a progreso o conservadurismo. Podrían quizá asociar alguna de los matices del concepto hacia uno u otro lado, igual que lo hacemos con respecto a internet, pero no al concepto en si de forma generalizada.

    Ahora bien, el punto de ruptura llega con el propio desmembramiento del antiguo régimen. Tras la Revolución Francesa empieza el proceso de rupturismo con respecto al feudalismo y el antiguo régimen a lo largo y ancho de Europa. Ahí tenemos la siguiente tesitura. Al haber sido una ruptura total el progresismo quiere separarse definitivamente de todo lo que representa el antiguo régimen y el conservadurismo mantenerlo. Y dentro de ese antiguo régimen hay un concepto que ha permeado en todas las estructuras de poder y en la sociedad misma, el cristianismo.

    Es por ello que podemos ver como desde ese punto los conservadores se aferran al cristianismo como parte del antiguo régimen que quieren conservar, o si se prefiere, como parte de las tradiciones heredadas de ese régimen, y los progresistas, que rompen de manera total con los conservadores, quieren deshacerse de todas esas tradiciones, entre las que incluyen al cristianismo.

    No obstante, aunque los conservadores quieran aferrarse al cristianismo, como ya comentamos anteriormente, el cristianismo no es algo que se aferre bien al conservadurismo.

    Imaginemos que llega una nueva ola progresista radical (ahora si de radicalismo), en este caso, el progreso para estas personas sería ir hacia un nuevo estadio de la humanidad en la que convivamos en armonía total con la naturaleza, prescindiendo de toda la tecnología anterior, y por ejemplo quisieran romper completamente con internet. Si a esta gente, que quiere ir hacia un nuevo progreso se convirtieran en los progresistas, ahora los conservadores serían aquellos que quisieran defender por ejemplo internet. Pero internet, hasta antes de ellos era un concepto aproximadamente neutro, no tenías que ser progresista o conservador para defender internet y sus bondades. 

    Este es un tema complejo, ya que objetivamente el cristianismo se mezcló con estructuras de poder y con pensamiento muy conservador de forma generalizada.

    Los ciudadanos medios de a pie, generalmente no tienen tiempo y/o voluntad suficiente para profundizar en temas complejos como este, por lo que prefieren ''comprar'' packs ideológicos completos o semi completos. Sea, cristianismo equivale a conservador, ateísmo o agnosticismo equivale a progresista. Y no entran en la complejidad de la teología completa. Además hay que sumar que a causa de esto, quienes más se acercan a la Iglesia, a formarla y desarrollarla desde dentro, son por tanto en cada vez más proporción estadística, los conservadores. Haciendo que la Iglesia, a la que se supone que representa el cristianismo sea más conservadora en formas y declaraciones, alejando cada vez más a los progresistas.

    Y he aquí el problema, si eres cristiano de conciencia, tanto progresista o conservador, no te gustaría que solo una parte de la población conociera tu mensaje o tus preceptos filosófico-teológicos mientras otra parte se aleja cada vez más. Al igual que no te gustaría que a día de hoy solo parte de la población quisiera conservar internet mientras otra abogara por eliminarlo. Te gustaría que el cristianismo fuera un concepto alejado del conservadurismo o del progresismo, de la política, y he ahí la clave.

    A partir de ese último punto es de donde extraigo mi conclusión. Los conservadores que no sean cristianos por sus convicciones religiosas sino por sus convicciones políticas no estarán interesados en que todo el mundo quiera ser cristiano, sino en que todo el mundo quiera ser conservador a través del cristianismo, por lo tanto no serán buenos cristianos, sino buenos conservadores. Mientras que los cristianos por convicción religiosa y no por convicción política estará interesado en que todo el mundo quiera ser cristiano independientemente de si es conservador o progresista. 
    
    Es por ello, que bajo mi punto de vista, lo que debería solicitar ese grupo de personas es que la religión desaparezca de los espacios políticos, que rechacen de pleno a partidos que se llamen cristianos o católicos, y que no quieran ver asociada la religión a la política, ya que inequívocamente esto producirá un rechazo en los oponentes políticos, así como rechazar a los cristianos que no son cristianos por convicciones religiosas sino políticas y expulsarlos de sus estructuras internas.    

    Mi conclusión final por tanto, es que si la religión quiere sobrevivir a la caída del antiguo régimen debe abandonar la politica y rechazar a la política que quiere hacer religión. Sino, estará abocada a dar sus últimos retazos al paso del tiempo y quedar olvidada para dar paso a nuevas cosas. No olvidemos que desde que nació el cristianismo y se impuso como nueva idea afianzada en la sociedad pasaron unos trescientos años, y que desde la Revolución Francesa hasta ahora han pasado más de doscientos treinta años. Si la cosa no cambia, y la Iglesia como comunidad no cambia de forma, puede que la historia se repita y nos encontremos ante los últimos setenta años de relevancia en Europa del cristianismo.

    Como siempre, agradecer que hayas leído mis reflexiones y conclusiones. Para concluir como con el resto de mis escritos,
    
    Y tú, ¿qué piensas?

    \includegraphics{Imagenes/FirmaMartin.png}
    