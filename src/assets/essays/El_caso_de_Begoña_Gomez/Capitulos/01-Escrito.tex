\center{\textbf{\LARGE{El caso de Begoña Gomez}}}\label{cap:Respuesta}

\justifying
    
    Normalmente no suelo tratar temas de esta índole en mis escritos, no obstante, este caso ha sido pedido directamente por un muy buen amigo que ve como en sus grupos de influencia las opiniones más sonadas carecen de rigor y se basan en la pura propaganda ideológica, así que haré todo lo posible para poder contraponer esa propaganda con un análisis más profundo del tema.

    Intentaré buscar siempre referencias directas a fuentes finales, en vez de a fuentes que vengan de algún medio de comunicación que siga una linea editorial de un partido o correiente ideológica u otra.

    Cabe aclarar, que aunque no suelo posicionarme de forma directa a favor o en contra de un personaje público en ningún escrito, en este caso no me quedará más opción que posicionarme. \textit{Pedro Sánchez} es una persona que está muy muy lejos de mi agrado, y como mi círculo cercano sabe, considero que debería haber dimitido hace tiempo. Esto se debe a múltiples motivos, pero ninguno relacionado con su esposa. 
    
    El que en mi opinión, el que debería haber sido el mayor catalizador de esta dimisión, es el caso de las amnistías, que si bien han resultado ser una herramienta tremendamente útil a la hora de combatir el independentismo, como bien avalan los resultados \cite{eleccionesCatalanas}, resultan completamente inmorales y deberían haber sido un arma de doble disparo; uno para combatir el independentismo, pero el segundo debería haber causado la caída de quien disparó por cometer un acto tan inmoral.

    Por otro lado, \textit{Koldo García} era la mano derecha de \textit{José Luis Ábalos}, siendo este la mano derecha del presidente hasta el año 2021. Cuando estás en política, tu equipo de confianza lo es todo, debes tener una confianza plena y absoluta en esas personas, y por ende, debes cerciorarte de que sus respectivos equipos (al menos la parte más cercana) son dignos de confianza también. \textit{Koldo García} es un personaje público de segunda fila, que venía de ser condenado en dos ocasiones por agresiones físicas (indultado de la pena de poco más de 2 años y del impago de una multa de \textit{100.000} pesetas (algo más de \textit{1.200} \euro{} a día de hoy) por el gobierno de \textit{José María Aznar} \cite{indultoKoldo}), además de haber agredido a un menor de edad junto con un policía nacional a un joven con una camiseta que ponía \textit{Independentzia} en un bar de Navarra \cite{sentenciaKoldo}. También ha trabajado en un prostíbulo, que si bien no tiene porque ser un descalificante, no parece un oficio para alguien que aspire a ser mano derecha de la mano derecha de un presidente de gobierno. Además pudiendo argumentar sanchistas que el no conocía o no aprobaba a este individuo, sabemos bien que no es cierto por las menciones directas que a el le ha hecho el presidente del gobierno \cite{facebookPresidente}.

    Aclarada mi posición con respecto al presidente y el poco aprecio que tengo a su figura política, pasaré a exponer mi argumentación de por qué el caso mediático de \textit{Begoña Gomez} tiene una altisima probablidad de tratarse de un \textbf{\textit{lawfare}}\cite{lawfare} de manual que será estudiado en el futuro, como ya lo fue el del caso de \textit{Mónica Oltra} \cite{monicaOltra}, siendo ese último, a mi parecer, mucho más fundamentado para realizar una investigación que el de la mujer del presidente.

    Toda la polémica empieza hace bastante tiempo, debido a que la figura política del presidente está en el foco de atención (como es entendible) y tiene una gran cantidad de enemigos políticos, con mucho peso en el ámbito civil y legal. Bajo mi punto de vista, el verdadero punto de inicio del \textit{lawfare} comenzaría a raiz de las elecciones generales del \textit{23 de julio de 2023}, donde la población votó mayoritariamente en contra de una coalición entre \textit{PP} y \textit{Vox}, aún habiendo sido este primero el que obtuvo una mayoría de votos. 
    
    Tras ese punto empezó lo que podríamos denominar como negociaciones agresivas entre partidos que pudieran formar una coalicción alternativa, resultando en el, si se me permite la expresión, \textit{frankeinsteiniano} gobierno de coalición que tenemos a día de hoy, con partidos con ideologías radicalmente enfrentadas, como pueden ser \textit{PNV} o \textit{Junts per Catalunya} (de ideología conservadora regionalista nacionalista independentista) con partidos como \textit{Esquerra republicana} (de ideología progresita socialdemocrata regionalista nacionalista independentista) o \textit{Sumar} y posteriormente \textit{Podemos} (que podríamos calificar como una malgama de partidos en coalición interna de \textit{izquierda indefinida} en términos de \textit{Gustavo Bueno}\cite{izquierdaIndefinida}), todos ellos bajo el liderazgo del \textit{PSOE}.

    Bajo este contexto de negociaciones agresivas entran en juego el punto ya explicado de las inmorales amnistías, a fin de conseguir un gobierno ``relativamente`` estable. Y es cuando llega el que a mi parecer es el verdadero punto de inflexión, cuando \textit{Jose María Aznar} desde su fundación de lobby \textit{FAES} lanza un mensaje llamando a la insurrección civil; \textbf{\textit{¿Qué se puede hacer? Pues el que pueda hablar, que hable. El que pueda hacer, que haga. El que pueda aportar, que aporte. El que se pueda mover, que se mueva. El que pueda intentar…}}\cite{aznarInstando}.

    Ahora tras esta introducción es cuando entra el caso que atañe. El sindicato de \textit{Manuel Bernad} (ex-político de \textit{Frente Nacional} con ideología neo-franquista nacional-católica \cite{miguelBernad}) presenta el día 9 de abril, la infame (desde un punto de vista jurídico por forma) ``denuncia`` \cite{denuncia} redactada a fecha 8 de abril. En esta denuncia se querella el señor \textit{Manuel Bernard} contra \textit{Begoña Gómez}, siguiendo de un \textit{esposa del presidente del gobierno} (luego aclararé porque esto es de importancia), por presulto delito de tráfico de influencias a tenor de lo preceptuado el artículo 429 del código penal \cite{articulo429}.

    En esta denuncia se vierten una serie de acusaciones relacionadas especialmente con el grupo \textit{Globalia} con intermediario \textit{Victor Aldama} de este supuesto tráfico de influencias, así como de la relación entre la acusada y \textit{Carlos Barrabés}, alegando relaciones de amistad entre todas las partes y que dieron como presunto resultado dos cosas. 
    
    La primera es la adjudicación de un rescate a \textit{Globalia} de 615 millones de euros según la denuncia, y como este, bajo la empresa \textit{Air Europa}, pacto pagar 40 mil euros al año al \textit{Africa Center}, del cual la denunciada fue directora ejecutiva durante el periodo del 2018 al 2022. Además de supuestamente pactar a pagar 15 mil euros al año en vuelos de primera clase para la denunciada y su equipo.

    La segunda es en relación a \textit{Carlos Barrabés}, al cual supuestamente se le adjudicaron licitaciones públicas por importe de 10 millones de euros, en base a las recomendaciones o avales de la denunciada.

    El colofón a esta denuncia es la inclusión de un \textit{Notitia Criminis}, a modo de fuente de pruebas de las anteriores acusaciones. Tratandose en este caso literalmente de ocho recortes de prensa de \textit{El confidencial}, \textit{Voz Populi}, \textit{Libertad Digital}, \textit{ES. Diario} y \textit{The Objetive}. Entre los que se incluye una noticia publicada por \textit{The Objetive} que reza ``\textit{El Gobierno ocultó el importe de una subvención a Begoña Gómez}`` \cite{articuloObjective}, con una foto de la mujer del presidente y el mismo. Dicho articulo fue desmentido un día después, ya que haciendo una mínima investigación periodística, se podía descubrir de que no se trataba de \textit{Begoña Gómez} la mujer del presidente, sino una mujer de Cantabria que regentaba un negocio, y que por cierto, le fue denegada \cite{buloBegoña}.

    Empecemos por partes. Con respecto al caso \textit{Air Europa}, se trata de una compañía que fue rescatada durante la pandemia. ¿Puede esto deberse a tráfico de influencias?. Quizá si fuese una empresa rescatada aislada en su sector podríamos interpretarlo así. No obstante, teniendo en cuenta que se trata de una compañía aérea y que todas ellas fueron rescatadas durante la pandemia, y que el importe del rescate se hizo en base a su facturación como el resto de rescates a compañías aéreas, parece que no tiene sentido relacionar ambos casos. Por lo que por ese lado podemos descartar, a priori, el trafico de influencias.

    Con respecto a los pagos a la fundación \textit{Africa Center}, se trata de un acuerdo de colaboración entre \textit{Air Europa} y una fundación del \textit{Instituto de Empresa}, y que ha contado en el año 2022 con 5800 donantes y en el 2023 con 6300 \cite{ieColabora}. Podría argumentarse que aunque la fundación recibe muchos donantes, el caso concreto de \textit{Air Europa} es especial, ya que ha sido rescatada y por tanto, haciendo un mortal con tirabuzón dialéctico-mental si la mujer del presidente se ha reunido con ellos, se trata de una trama de corrupción por tráfico de influencias. No obstante, si fuera el caso, entiendo que también debería pasar lo mismo con la \textit{Fundación mensajeros de la paz}, \textit{Asociación de empleados de aeronáutica}, y en general todo lo relacionado con el programa \textit{Somos Solidarios} de \textit{Air Europa} con el que ha estado realizando donaciones a causas similares y que le valieron el ganar el sello de \textit{Empresa Solidaria} otorgada por \textit{la ONGD Manos Unidas} \cite{manosUnidasAirEuropa}, en la que, por otro lado como es normal, las empresas beneficiarias de las ayudas utilizan esas ayudas para desplazar a sus respectivas juntas directivas a los lugares que conciernen.

    Por otro lado tenemos las acusaciones con respecto a \textit{Carlos Barrabés}. Fundador y Owner del grupo Barrabés, una consultora con oficinas en Madrid, Finlandia y Emiratos Árabes Unidos, que ha recibido en múltiples ocasiones concesiones públicas \cite{consultora}. En este caso, parece medianamente evidente que la denunciada tiene o tenía cierta relación de amistad o cordialidad con el señor \textit{Barrabés}, y esto, hizo que la denunciada redactara una carta de recomendación al empresario para la adjudicación de una concesión publica en concreto para la creación de un máster en captación de fondos \cite{recomendacion}. Aún siendo esto bajo mi punto de vista un error político por la posición que ostentaba en ese momento la denunciada, hay que destacar que se trata de una entre 32 cartas de recomendación que se recibieron para esta concesión, y que la cuantía que se le atribuye en la denuncia a esta carta, no es la del máster, sino toda la que ha recibido en su carrera \textit{Carlos Barrabés}, lo cual es un auténtico despropósito.

    Pero, entonces, porqué se sigue hablando de este caso, si claramente se trata de una denuncia sin fundamento, como por cierto, así ha avalado la guardia civil en un informe de 160 páginas en el que concluía que no hay indicios de delito en la actuación de la mujer del presidente del Gobierno.

    La clave está en lo que decíamos de la frase pronunciada por el ex-presidente del gobierno llamando a la insurrección. Esta frase fue escuchada por el juez que admitió a trámite esta denuncia, que en cualquier universidad de derecho le habría valido un suspenso a cualquier alumno. El juez \textit{Juan Carlos Peinado}.

    Este juez se trata de un juez ya conocido por su parcialidad y sesgo ideológico extremo escorado a la derecha. Este juez accedió al juzgado de Madrid sustituyendo a su predecesora, que entró en el gobierno de \textit{Cifuentes} como directora general de Justicia y Seguridad de la Comunidad de Madrid. Es ni más ni menos que el padre de \textit{Patricia Peinado}, concejala del Partido Popular en el ayuntamiento de Pozuelo de Alarcón. Este hombre es el juez responsable de la imputación de 12 periodistas por informar del caso de los CDR, que por cierto, terminó archivandose no sin antes haber provocado un inmenso dolor y suplicio a las familias de esos periodistas y a ellos mismos por realizar su trabajo. Además, \textit{Manos Limpias} ya ha acudido a el en otras ocasiones, como cuando presentaron la denuncia contra \textit{Pablo Soto} y \textit{Guillermo Zapata} por unos twits ofensivos. En resumen, no es un juez que podamos decir imparcial de la justicia. Y ojo, no digo que tal cosa exista, todos tenemos nuestras ideas políticas y nuestros sesgos, no obstante, la gente seria y justa trata de ser lo más objetiva posible. Este juez no.

    El caso ha vuelto a saltar a la palestra, ya que en un procedimiento extremadamente irregular, \textit{Juan Carlos Peinado} la había calificado como investigada el 16 de abril, es decir, tan solo 7 días después de la presentación de la denuncia. Tras eso, el juez exigió que no se le notificara hasta la semana siguiente, el mismo día en el que el presidente anunció su periodo de reflexión. Cabe destacar que en el proceso de una investigación es paso necesario la llamada a declaración durante el análisis de admisibilidad previo a la investigación, y este no se produjo \cite{etapasInvestigacion}. Es decir, estamos ante un juez que se saltó los pasos de análisis de admisibilidad, admitiendo a trámite una denuncia altamente manipulada y burdamente realizada en forma, y que se saltó los procedimientos legales hasta poder otorgar el estatus de investigada a la denunciada para poder realizar un daño político.

    Cuando se conoció el caso de la denuncia, los medios de comunicación, incluso los más de derechas, no podían ni imaginar que la denunciada tuviera el estatus de investigada, ya que se trata de una cosa tan irregular, y bajo mi humilde opinión tan burda, que no era ni concebible. Por lo que todos los medios especulaban si finalmente se investigaría o no a la denunciada, pero la realidad ha venido para golpear y decirnos que por muy rocambolesco que sea, ya estaba siendo tratada de investigada en los juzgados de Madrid, a causa de un juez, al que no me atrevo a calificar directamente como corrupto, pero si como claramente sesgado ideológicamente, y movido a la insurrección provocando un escandaloso caso de lawfare. Y esto ha provocado que los que quieren atacar al gobierno y concretamente al presidente, puedan hacerlo con la boca llena con palabras que deberían tener un significado muy importante como es el de ``investigada`` pero que por culpa de cosas como esta se está despojando de valor real.

    Ahora cabe hacerse la pregunta, ¿cómo podría haberse evitado esto? ¿existen alternativas para evitar el lawfare? ¿existen modelos políticos diferentes para que un jefe de estado no tenga que preocuparse por el tráfico de influencias o sea una preocupación mínima?

    Yo tengo mis ideas y opiniones al respecto, no obstante, para concluir como con el resto de mis escritos,

    Y tú, ¿qué piensas?
