\begin{center}
  \textbf{\LARGE{Valencia inundada}}
\end{center}
\label{cap:Respuesta}

\justifying
   
   \section*{Preámbulo del escrito}

   
        Normalmente no escribo de cosas que estén cercanas en el tiempo, sino de hechos que ya hayan podido ser analizados con cierta perspectiva temporal. No obstante, una persona muy cercana a mi me hizo llegar mensajes propagados por \textit{influencers} sobre los hechos que atañen al escrito, y me los hizo llegar no solo a mi, sino que los propagó a más gente cercana comentando que sus pensamientos eran muy cercanos al mensaje enviado, resultando \textit{ciertamente razonable} para el resto de personas cercanas que recibieron el mensaje. 
        
        Esta persona en concreto estoy bastante seguro de que posee altas capacidades intelectuales, por lo que el tipo de mensajes que me transmitió y el contenido de los mismos me llevó a dedicar tiempo de investigación y reflexión sobre los hechos. Si para alguien con tales capacidades, y que además, de forma general no se interesa por la política llega a esas conclusiones, o son verdades evidentes, o algo está fallando, y quiero averiguar que es. Puesto que no tenía en ese momento una opinión clara sobre el tema, y no quería responder precipitadamente consideré necesario realizar este escrito.

    \section*{Escrito}
    
        El martes veintinueve de octubre se produjo en \textit{España} uno de los sucesos más catastróficos de las ultimas décadas en el país. Desde mil novecientos noventa y seis, no se sufría a nivel nacional, una catástrofe natural tan cuantiosa en fallecidos, la tormenta en \textit{Biescas} de aquel agosto fue catastrófica, y aún así dejo menos de cien muertos; cifra ya superada con creces y subiendo a día de hoy en el caso de la última catástrofe que se está viviendo en la \textit{Comunitat Valenciana}. En daños materiales quizá lo más cercano que tengamos sea la erupción del \textit{Tajogaite} de \textit{La Palma} en dos mil veintiuno, que dejó una persona fallecida y más de ochocientos millones de euros en daños.
    
        La depresión aislada en niveles altos (\textit{DANA} en adelante) que ha sufrido el país y especialmente la zona de Valencia tiene muy pocos precedentes, y ninguno en nuestra historia reciente del país. Es probable que deba actualizar este escrito en el futuro, ya que los hechos se siguen sucediendo, pero, la pregunta importante es. ¿\textbf{Qué} se está haciendo para ayudar a la gente que necesita ayuda? ¿De \textbf{quién} o quienes es la culpa de que no se esté ayudando a la gente que lo necesita? ¿\textbf{Cómo} debería haberse actuado para que esto no estuviera pasando así?
    
        Lo primero es entender que pasó y como pasó. Para esto utilizaré principalmente un articulo realizado por \textit{Eduardo Robaina} del portal \textit{Climatica.coop}\cite{climatica}.
    
        El primer momento que se empezó a hablar sobre la \textit{DANA} se trató de un aviso \cite{aemetPrimer} por la Agencia Estatal de Meteorología (\textit{AEMET} en adelante). Este es un órgano dependiente del \textit{Gobierno Central} mediante la \textit{Secretaría de Estado de Medio Ambiente} y en última instancia del \textit{Ministerio para la Transición Ecológica}
        
        ''\textit{En los próximos días, un embolsamiento de aire frío quedará aislado de la circulación general, dando lugar a la formación de una dana. Se acercará a nuestro entorno, con lluvias y chubascos a partir del sábado, aunque con incertidumbre aún sobre las zonas con mayor probabilidad.}''
    
        En ese momento, ninguna de las autoridades le dio mayor importancia, ni estatal ni regional. No obstante el jueves veinticuatro la alerta lanzada por esta agencia se elevó \cite{aemetSegundo} y ya se habló por primera vez de las zonas que podrían verse afectadas. 
    
        ''\textit{El viernes, un frente dejará precipitaciones en amplias zonas de España. A lo largo de los días siguientes, una dana se acercará a nuestro entorno, con mucha incertidumbre aún sobre su posición final. Ahora mismo, sería el este peninsular la zona que más lluvia recibiría.}''
    
        En ese momento, ninguna administración dijo nada al respecto. El propio veinticuatro por la noche, la empresa \textit{Meteored} avisó \cite{meteoredPrimer} de la envergadura de lo que se venía. Ya no hablamos solo de una agencia pública controlada por el \textit{Gobierno Central}, sino de una empresa privada especializada en meteorología.
    
        ''\textit{La DANA que nos afectará en unos días no es una más, a los meteorólogos/as nos está quitando el sueño. Te contamos qué la hace especial y por qué es tan complicado pronosticar sus posibles consecuencias.}''
    
        Hasta este momento, a ninguna administración, regional o estatal se le debería achacar nada. No había ningún aviso formal de ninguna agencia estatal. Podríamos decir que se deberían haber tenido en cuenta los avisos de \textit{Meteored}, y sería correcto, pero mucha gente podría argumentar que es una agencia que ya se ha equivocado mucho en el pasado y ha levantado alertas cuando no debería, y es algo que ellos mismos reconocen \cite{meteoredFallos}.
    
        No fue hasta el viernes veinticinco mismo cuando se emitieron, con más datos sobre la magnitud y precisión sobre la mesa, el primer aviso formal de una agencia estatal, la \textit{AEMET}, avisando de lo que se vendría el sábado \cite{informeAEMET}. Ya se avisaba en ese momento de que en el litoral se iban a acumular hasta diez litros de agua por metro cuadrado (dato que enmudece con los seiscientos treinta litros por metro cuadrado que se llegaron a registrar, la mayor cifra registrada en el siglo XXI en la \textit{Comunitat Valenciana}).
    
        Si bien es cierto que en el propio escrito mencionaban que aún existía cierta incertidumbre ''\textit{... Todavía existe gran incertidumbre sobre la localización exacta de los mayores acumulados, pero es posible que en puntos de la vertiente mediterránea se den chubascos y tormentas fuertes, o muy fuertes, y localmente persistentes, más probables durante el martes 29...}'', se trata de una alerta formal, que debería levantar inmediatamente una alerta en las administraciones para empezar a prepararse ante posibles situaciones complicadas.
    
        Hay que comentar también que algunos meteorólogos ya estaban avisando de la posibilidad de que esta \textit{DANA} en particular fuera muy catastrófica en los próximos cinco días \cite{avisoMeteorologo}. Ante esta alerta, la mayoría de las respuestas eran similares a las siguientes:
    
        \begin{itemize}
            \item ''\textit{Pero va llover o no? En Córdoba, siempre lo mismo muchos datos y alertas y luego na}''
            \item ''\textit{Buff si, lo mismo que la temporada de huracanes de este año que, según farsantes climáticos como tú, iba a ser nunca vista. El guión de la farsa climática exige exageraciones y tergiversaciones diarias. Aunque como con la lotería, si juegas cada día, alguna vez acertarás.}''
        \end{itemize}
    
        Hay que destacar que la mayoría de los comentarios de ese día han sido borrados por los autores.
    
        Tras esto, el propio lunes se siguieron más avisos formales de la \textit{AEMET} Que ya focalizaban más la zona y aumentaban las probabilidades a más del setenta por ciento y focalizando hacia la zona de la \textit{Comunitat Valenciana} \cite{aemetSiguiente}\cite{aemetDia}.
    
        En este punto, algunas partes de la administración regional empezaron a tomar ciertas medidas autónomas, por ejemplo cancelar las clases en colegios y universidades. Estas medidas se realizaron tras las reuniones celebradas en los centros de coordinación de Emergencias 112 a la que asistieron representantes de las consejerías de Presidencia regional.
    
        Fue el martes veintinueve a primera hora por la mañana, cuando la \textit{AEMET} puso el nivel de alerta máximo en la \textit{Comunitat Valenciana}, esta vez, haciendo llamados directos al \textit{Gobierno Regional} por la red social \textit{X} \cite{aemetRojo}, y por la tarde un comunicado especial \cite{aemetComunicado}. Cabe destacar que ya en este comunicado avisaban de que esta \textit{DANA} duraría hasta el jueves treinta y uno, y que no había previsiones de que fuera a amainar pronto.
    
        Ese mismo día, la \textit{Confederación Hidrográfica del Júcar} convocó la reunión de emergencia a las diez de la mañana para tratar sobre el tema junto con la \textit{Delegación de Gobierno de la Comunitat Valenciana}, la delegación de \textit{AEMET} valenciana, la \textit{Dirección General de Tráfico}, la \textit{Guardia Civil}, la \textit{UME}, \textit{ADIF}, \textit{Renfe} y \textit{Demarcación de Carreteras}\cite{chjPost}.
    
        En ese momento ya había localidades como \textit{Carlet} donde se habían acumulado más de ciento veinte litros por metro cuadrado, unas medidas que ya excedían por mucho la situación de alerta roja de la \textit{AEMET}\cite{chjPostPrevio}.
    
        En estos momentos en los que la situación ya era gravísima y había grandes indicios de que las cosas iban a ir a peor, lejos de suspender la actividad productiva en la región por decreto y obligar a empresas a no trabajar esa jornada para que los ciudadanos pudieran resguardarse en sus casas o lugares seguros, lejos de proporcionar consejos sobre como evitar los desastres y actuaciones a seguir; el presidente de la \textit{Generalitat Valenciana} anunciaba en comunicado de prensa oficial lo siguiente a las trece horas:
    
        ''\textit{Según la previsión el temporal se desplaza hacia la serranía de cuenca en estos momentos, por lo que se espera que en torno a las 18.00 horas disminuya su intensidad en todo el resto de la comunidad valenciana}'' \cite{mazonAnuncio}
    
        Pocas horas después, esta publicación era eliminada por el \textit{govern de la Generalitat Valenciana}, aunque ya había sido descargado para evitar que se borrasen las pruebas de ese mensaje.
    
        Hay que destacar también una cosa importante. La \textit{Comunitat Valenciana} tiene un secretario autonómico de \textit{Seguridad y Emergencias}, que es además el director de la \textit{Agencia Valenciana de Seguridad y Respuesta a las Emergencias}, \textit{Emilio Argüeso Torres}. Ante esta situación, uno cabría esperar que este hombre estuviera gestionando la situación, levantando la alerta si hiciera falta, y preparando y coordinando la respuesta a la emergencia, como indica su cargo. No obstante, a las trece horas estaba teniendo una reunión junto al jefe del \textit{Servicio de Espectáculos Públicos, Actividades Recreativas y Festejos Taurinos}, en la que ni muchísimo menos se habló del tema verdaderamente importante en ese momento para su puesto.
    
        Tras esto los hechos empezaron a precipitarse, las noticias de los primeros muertos y desaparecidos empezaron a llegar. En este punto desde el \textit{govern de la Generalitat Valenciana} se empezó a sopesar si elevar la alerta y pedir ayuda al \textit{Gobierno Central}, los únicos habilitados para desplegar a la \textit{Unidad Militar de Emergencias} (en adelante \textit{UME}) que depende operativamente del \textit{43 Grupo de Fuerzas Aéreas} y el \textit{Batallón de Helicópteros de Emergencia II}, que pertenecen al \textit{Ejército del Aire y de Tierra} respectivamente. 
        
        A las veinte horas, sonaron en los teléfonos de los ciudadanos de la comunidad el aviso de protección civil llamando a evitar desplazamientos. A esas horas, las jornadas laborales de la mayoría de trabajadores habían terminado, y muchos encontraron los desbordamientos en torno a las diecisiete volviendo a sus residencias. Algunos otros se vieron obligados a refugiarse en sus lugares de trabajo, y a otros les pilló en mitad de tránsito dentro en el trabajo. 
        
        A las veinte y treinta y seis el \textit{govern de la Generalitat Valenciana} solicitó formalmente al \textit{Gobierno Central} la ayuda de la \textit{UME}. En esos momentos ya se habían identificado casos de desaparecidos y muertos por decenas.
    
        Tras esto, el \textit{Gobierno Central} bajo mandato del ministro de política territorial y memoria democrática de España, \textit{Ángel Víctor Torres}, movilizó a mil cien efectivos de la \textit{UME}, setecientos cincuenta agentes de la \textit{Guardia Civil} y cien agentes de la \textit{Policía Nacional} \cite{movilizacion}.
    
        Y tras esto, se sucedió el horror de la noche y de los siguientes días. Muertos aparecen por decenas, sumando más de doscientos rápidamente. Daños materiales por valor estimado aproximado de más de mil millones, desaparecidos, muertos, familias rotas...
    
        Esta es la historia hasta este momento. Ahora hablemos de las responsabilidades y de los pensamientos generales.
    
        Empecemos por el principio. Hay que destacar algo importante, y que ha envejecido en estos momentos muy mal. El \textit{govern de la Generalitat Valenciana} desmanteló en febrero del dos mil veintitrés la \textit{Unidad Valenciana de Emergencias} (la \textit{UVE} en adelante). Este organismo tenía como objetivo ''\textit{hacer frente a las necesidades derivadas de la respuesta en emergencias, en los términos establecidos en la planificación de protección civil de la Comunitat Valenciana}'' \cite{creacionUVE}. Este organismo fue creado bajo el mandato de \textit{Ximo Puig}, el expresidente de la \textit{Generalitat Valenciana}, gobierno del \textit{Partido Socialista Obrero Español} (\textit{PSOE} en adelante). Este era el motivo principal para desmantelar los organismos creados por el expresidente por el nuevo presidente del \textit{Partido Popular} (en adelante \textit{PP}) \textit{Carlos Mazón}. 
        
        Esta decisión, para ser justos se tomó bajo la petición de quienes tenían las competencias en materia de emergencias, es decir, bajo la \textit{Conselleria de Justicia}, dirigida tras el acuerdo entre \textit{PP} y \textit{VOX} por \textit{Elisa Núñez} de \textit{VOX} en aquel momento y que abandonó la formación ultra conservadora para pasar a formar parte del \textit{PP} en dos mil veinticuatro cuando los pactos autonómicos saltaron por los aires.
    
        Este organismo podría haberse activado sin necesidad de elevar la alerta al \textit{Gobierno Central}, motivo principal por el cual \textit{Mazón} no realizó la alerta, ya que al ser este del \textit{PP}, mostraría debilidad ante el \textit{PSOE}. Pero al no existir, no se pudo desplegar esta ayuda. 
    
        Esta actuación de desmantelar instituciones como la \textit{UVE} que podrían haber salvado tantas vidas, de hacer caso omiso a las advertencias del \textit{Gobierno Central} a través de la \textit{AEMET}, de afirmar que amainaría la tormenta cuando todas las evidencias y expertos ya indicaban que iría a peor, y posteriores declaraciones que se mencionarán en este escrito más tarde; deberían ser más que suficiente para no solo exigir la dimisión de \textit{Mazón}, sino para abrir una macro-causa penal contra el por homicidio por omisión, homicidio imprudente o negligente o incluso homicidio doloso por omisión.
    
        Por otro lado, nos encontramos al \textit{Gobierno Central}. En este caso dirigido por el \textit{PSOE}. Si bien es cierto que a través de la \textit{AEMET} fueron proporcionando información y recomendaciones que deberían haberse seguido por las competencias autonómicas, estaba bajo su potestad a través del \textit{Ministerio de Interior} dirigido por \textit{Fernando Grande-Marlaska} el activar el nivel tres de emergencia nacional. 
    
        Esta medida recogida en la Ley 17/2015 \cite{leyProteccion} ''\textit{puede ser tomada a petición de las Comunidades Autónomas o de los Delegados del Gobierno en las mismas.}'' Con ella permitiría al Ministerio de interior ''\textit{su dirección, que comprenderá la ordenación y coordinación de las actuaciones y la gestión de todos los recursos estatales, autonómicos y locales del ámbito territorial afectado}''. Con esto el \textit{Gobierno Central} podría, en primer lugar, haber decretado el cese de la actividad productiva total durante el día de la catástrofe para evitar que la gente se desplazase a sus trabajos, haber desplegado antes a los miembros de la \textit{UME}, y en última instancia haber salvado vidas.
    
        Esta medida nunca ha sido aplicada y representa, de manera evidente, una acción extremadamente represiva hacia las libertades ciudadanas. Además, supone una forma de eludir la autoridad de las competencias de la \textit{Generalitat Valenciana}, cuyo gobierno tiene una postura ideológica y de gestión opuesta al \textit{Gobierno Central}. Por tanto, resulta comprensible que, especialmente tras los conflictos legales surgidos durante la pandemia cuando el \textit{Gobierno Central} fue judicializado por restringir ilegalmente las libertades de los españoles, el \textit{Gobierno Central} decida no implementar esta medida.
    
        ¿Comprensible?, por supuesto, tomar esta medida les hubiera costado una cantidad de críticas y perdida de votos superior a si no la hubieran tomado. Alguien en el \textit{Gobierno Central} hizo los números para ver cómo podían perder menos votos. ¿Justificable?, es más complicado. El \textit{Gobierno Central} se supone que debe velar por el bienestar de los ciudadanos, por su seguridad y por su integridad. Esto implica, bajo mi punto de vista, que el cálculo que realizaron no debió haber sido sobre cuantos votos iban a perder, sino sobre cuantas vidas iban a salvar. Es evidente que debido a la crispación política que vivimos, hiciera lo que hiciera el gobierno iba a estar mal, bueno, pues como se van a quejar de todas formas, al menos salva vidas.

        Algo a destacar, y que es absolutamente despreciable por parte del \textit{Gobierno Central} es que no se cancelara el pleno, una vez ya solicitada la ayuda por parte del \textit{govern de la Generalitat}, en el que se votaba una medida referente a la dirección de \textit{RTVE}, y que claramente tenía tintes partidistas. No se canceló ya que esta convalidación del Real Decreto Ley al que se hacía referencia podría haberse visto paralizada en caso de no haberse votado ese día. Seguramente, a nivel de partido para el \textit{PSOE} y sus aliados de coalición liderados por \textit{SUMAR} sería algo de vital importancia (cosa que es cuestionable y muy polémica de por sí), pero al haberle dado prioridad a esto, no han mostrado unidad en apoyo a los afectados. Para más escarnio, si decidieron suspender por ese motivo la sesión de control al gobierno, de realización inmediatamente posterior; ya que esta sí que era utilizar el valioso tiempo de los integrantes del \textit{Gobierno Central}, en algo que no fuera la gestión de la crisis.
        
        Volviendo a lo anterior, otra medida que se podría haber tomado es la de haber decretado un estado de alarma, similar al que se decretó durante la pandemia. Esto también habría puesto a disposición del \textit{Gobierno Central} las herramientas necesarias para dirigir la situación, sorteando la administración regional. No obstante el caso aquí es similar, y además con más antecedentes del desgaste político que puede producir una medida de este calibre, antecedentes por los cuales hasta día de hoy se sigue arrastrando frases de que vivimos bajo un estado policial y un sistema totalitario \cite{totalitarismo}. 
        
        Con respecto a este último punto, hay que destacar que el propio \textit{PP} de la \textit{Comunitat Valenciana} solicitó en un \textit{tweet} que se decretara este estado de alarma \cite{pidenEstadoAlarma}, y este \textit{tweet} fue inmediatamente eliminado a los pocos minutos, ya que esto sería asumir públicamente que su gestión no es suficientemente válida para gestionar la crisis. 
    
        Añadamos también las intervenciones de las distintas entidades políticas. Por un lado tenemos a \textit{Mazón} que además de todo lo dicho, sus últimas aportaciones son contra el \textit{Gobierno Central} y las entidades como la \textit{AEMET} y la \textit{ Confederación Hidrográfica del Júcar }. Además también hay que destacar que desde el primer momento fue reacio a que la \textit{UME} pudiese operar en el territorio, dejándoles solo operar en un primer momento en las zonas de \textit{Utiel} y \textit{Requena} y no autorizando su intervención en más lugares donde evidentemente eran necesarios. La propia Ministra de defensa \textit{Margarita Robles} hacía declaraciones públicas hablando de la frustración que supuso para ella personalmente y para los altos mandos del ejercito el no poder desplegarse en más lugares por la negativa del \textit{govern de la Generalitat} de \textit{Mazón}.
    
        En estas mismas declaraciones la Ministra confirmaba que la dirección de la emergencia, en ausencia de un estado de alarma o de un nivel tres de emergencia nacional, corresponde al gobierno regional y no al central\cite{laSextaRobles}. Cosa que así se recoge en el BOE\cite{leyProteccionDos}. Ante estas declaraciones, la portavoz del \textit{govern de la Generalitat}, \textit{Ruth Merino} ex portavoz del semi-extinto \textit{Ciudadanos} que se pasó al \textit{PP}, calificó a la Ministra de ''\textit{intolerable la deslealtad}'', alegando que la ministra ''\textit{debe colaborar y dejar de entorpecer las labores de rescate y la ayuda humanitaria}'' \cite{vanguardiaRoblesMazon}.
    
        Dentro de todo esto hay que destacar una cosa importantísima que no debemos pasar por alto. La impecable y extremadamente célere efectividad de la \textit{UME}. Desde el primer momento que fueron autorizados a desplegarse, se pusieron a disposición del \textit{govern de la Generalitat} y empezaron a actuar cerca de mil doscientos activos de la \textit{UME} en las zonas autorizadas, y a las pocas horas ya estaban realizando labores de rescate de cadáveres y restablecimiento de suministros. En momentos como estos, y muy a pesar de quienes se opusieron desde la oposición a su creación por el ex presidente \textit{José Luis Rodríguez Zapatero}, demuestra porque es un cuerpo del ejército tan importante.
    
        Al respecto de esto, no paran de salir influencers como \textit{Marc Vidal} \cite{markVidal}, \textit{Ruben Gisbert} \cite{gisbert}, \textit{Wall Street Wolverine}\cite{wsw} y personas anónimas; quejandose del \textit{Gobierno Central} y de su falta de actuación, llegando a decir y cito textualmente:
        
        ''\textit{...¿Qué es cuando nos dicen que hay mil doscientos efectivos ya en la zona desplegados del ejercito de la UME o de no sé quién?...En España hay prácticamente ciento veinte mil soldados en activo ... pero, ¿sólo mil doscientos? ¿sólo mil doscientos en una zona que ahora mismo es una catástrofe absoluta de tipo humanitario? ... pero, mañana van a llegar quinientos más ha dicho la Ministra de defensa, ¿quinientos soldados más? ¿pero por qué no enviamos a treinta mil? y ponemos todo eso patas arriba y ayudamos a todo el mundo}...'' 
    
        Este mensaje ha llegado a más de trescientas mil personas a través de estos \textit{influencers}. \textit{Influencers} cuyo objetivo es desestabilizar lo máximo posible a cualquier gobierno que no sea prácticamente anarcocapitalista. Y este mensaje le llegará a gente que estará en lineas generales muy enfadada con el mismo. Bueno, estos mensajes son un problema tremendo, ya que son mensajes enviados desde el absoluto desconocimiento de como funciona la realidad.

        Para empezar, el primer punto, es que como ya hemos mencionado previamente, no es potestad del \textit{Gobierno Central} el desplegar al ejercito en una zona de catástrofe, a no ser que el gobierno regional lo solicite explicita y formalmente, o que el \textit{Gobierno Central} se salte las libertades de los ciudadanos declarando un estado de alarma o un estado de emergencia nacional de nivel tres (esto último es algo que desde mi punto de vista debería haber pasado). No puedes criticar a la vez al \textit{Gobierno Central} por no actuar, y a la vez criticarlo por querer actuar y no poder, y si lo haces, eres un manipulador o alguien falto de coherencia y de juicio cuestionable. 

        Por otro lado, la \textit{UME}, ese cuerpo tan criticado durante su creación, cuenta a momento de realización de este escrito, con tres mil doscientos veintiocho militares entre Cuadros de Mando y personal de Tropa y Marinería \cite{ume}, de los cuales el \textit{Tercer Batallón de Intervención} cuenta con quinientos veintisiete efectivos en el cuartel de \textit{Bétera}, en \textit{Valencia} \cite{umeValencia}. El siguiente batallón más cercano se trata del \textit{Cuarto Batallón de Intervención}, en el cuartel de \textit{Zaragoza}, con un total de cuatrocientos ochenta y tres militares \cite{umeZaragoza}. Estos fueron los primeros en llegar.

        El ejército, no funciona de forma que los soldados aparecen en la zona y ejercen el control y la ayuda que tengan que ejercer. Para empezar, el único cuerpo capacitado para el despliegue rápido en una zona muy alejada del cuartel de destino, esto es, en menos de veinticuatro horas; a día de hoy es la \textit{UME}. Por lo que en el mejor de los casos podrían desplegarse una proporción alta de los tres mil doscientos veintiocho militares, que recordemos que incluyen no solo a Tropa sino a otro tipo de personal. Actualmente hay desplegados más de dos mil, por lo que están ya cerca del límite de las capacidades operativas del cuerpo.

        Por otro lado, un despliegue no es lanzar a los soldados a la zona afectada, la logística es la parte más importante de un despliegue. El general y expresidente de los \textit{Estados Unidos de América} \textit{Dwight D. Eisenhower} ya dijo una frase al respecto: ''\textit{No encontrarás difícil demostrar que las batallas, las campañas e incluso las guerras se han ganado o perdido, principalmente, por la logística”}''. Mandar a treinta mil soldados como algunos sugieren, solo provocaría que las carreteras, ya semi-colapsadas, colapsaran por los convoyes militares, al cabo de las horas la escasez material se agravaría y los propios desplegados no tendrían que comer, beber o donde dormir; provocaría un desabastecimiento de las capacidades defensivas del estado en un nivel severo y durante un tiempo prolongado... Sencillamente, decir frases así, es no entender el ejército, para que sirve, y cómo funciona.

        La ayuda humanitaria es imprescindible en estos casos, y debe estar coordinada. Casos como el de \textit{Rubén Gisbert}, desplazándose a la \textit{Zona Cero} del desastre; aunque con la mejor y más filantrópica de las intenciones, provoca un efecto llamada de voluntarios externos a la \textit{Comunidad Autónoma} para llenar las redes, en la mayoría de los casos, con sus aportaciones ''completamente desinteresadas'' para sus cuentas de \textit{Instagram}, \textit{TikTok}, \textit{Twitter}... Y esto, provoca a su vez el colapso de las carreteras al ocupar estos voluntarios los pocos carriles despejados en muchos sitios, los fallos en la coordinación malgastando recursos en llegar a sitios por los que ya ha pasado gente, faltas de suministros nuevas al llegar gente sin preparación y sin nuevos suministros. Tanto es así que tanto el \textit{Gobierno Central}, bajo el \textit{Ministerio del Interior} como el \textit{govern de la Generalitat}, ya han emitido un comunicado pidiendo a los voluntarios que dejen de ir a las zonas afectadas para no colapsar las carreteras y que faciliten el trabajo a los profesionales\cite{voluntarios}.

    \section*{Conclusiones}

        Fuera de este escrito se han quedado una ingente cantidad de bulos y manipulaciones muy burdas asociados que he podido ir visualizando, desde bulos asociados al franquismo sobre presas destruidas y como estos hechos afectaron al desastre; hasta historias completamente sesgadas y racializadas para aprovechar a enviar un mensaje de odio xenófobo. Ya que considero que de forma generalizada, quien pueda leer este escrito ya entiende y discierne esos bulos y burdas manipulaciones (al menos eso espero).

        Creo, que tras haber dedicado un tiempo muy considerable a informarme, a contrastar diferentes fuentes y a reflexionar sobre el tema, puedo llegar a algunas conclusiones al respecto y formar una opinión. Es algo precipitado para mi gusto, por supuesto, y es probable que a la luz de nuevos datos pueda perfectamente cambiar de opinión con respecto a este tema. Pero, considero que una opinión al respecto de este tema, usando únicamente uno o dos titulares sueltos y un par de publicaciones de \textit{influencers} anti-sistema, conduce a conclusiones profundamente equivocadas y que solo incitan al odio y al desconocimiento.

        Mis conclusiones son las siguientes.

        Por un lado, la actuación del \textit{Gobierno Central}, con respecto a la gestión y sensibilidad de las sesiones parlamentarias realizadas tras la catástrofe, son deleznables y extremadamente ruines.

        Considero, que el \textit{Gobierno Central}, ante la falta de solicitud de ayuda del \textit{govern de la Generalitat Valenciana}, debería haber decretado el estado de alerta nacional de nivel tres y actuar directamente. Esto hubiera causado un gran revuelo a nivel nacional, y habría provocado que se acrecentara la visión sesgada, y bajo mi punto de vista falsa y errónea promovida por voces ultra conservadoras y anti-sistema, de que vivimos en un estado totalitario; pero habría salvado muchas vidas y eso debería haberle sopesado más que los votos que pudieran haber perdido.

        Bajo mi punto de vista, el presidente de la \textit{Generalitat}, \textit{Carlos Mazón Guixot}, por todo lo presentado, no es que debiera dimitir (que por supuesto), es que debería ser juzgado y asumir responsabilidades políticas y penales tras finalizar la crisis, en vez de ser respaldado por el presidente del partido \textit{Alberto Núñez Feijóo}, que actualmente se está dedicando a eximirle de responsabilidades y atribuírselas falsamente y aún a sabiendas, al \textit{Gobierno Central}.

        Finalmente, por un lado me asusta y me preocupa, como los mensajes de \textit{influencers} anti-sistema e incitadores al odio se extienden y utilizan el dolor de victimas para atacar indiscriminadamente al \textit{Gobierno Central}. Me aterroriza profundamente, que estos mensajes puedan calar en personas que tengan incluso altas capacidades intelectuales, aun siendo simplistas y tergiversadas. Me apena que siempre sean las mismas voces las que lanzan estos mensajes, y que sobre todo, las voces sensatas, formadas e informadas, lleguen a mucha menos gente que las contrapartidas. Ojalá eso pueda cambiar algún día, y esas voces sean las que suenen con fuerza y muevan los debates entre la población civil, los debates gubernamentales y las agendas políticas de los mismos gobernantes; y no las que lo hacen ahora.
        
        Como siempre, agradecer que hayas leído mis reflexiones y conclusiones. Para finalizar co<mo con el resto de mis escritos,
        
        Y tú, ¿qué piensas?
    
        \includegraphics{Imagenes/Firma.png}
            
        
        
    
        
    
        
        
        
    
         