\center{\textbf{\LARGE{Debate interno sobre el mal}}}\label{cap:Inicio}

\justifying

    Este escrito está un poco más alejado del resto de mis escritos que suelen discernir sobre política en general. No obstante, como en el resto de mis escritos, el principal fin de realizarlo es para poder realizar una investigación más exhaustiva y poder reflexionar más sobre algunos de los debates que he tenido y de los cuales me gustaría seguir profundizando.

    Algo de lo que estoy muy agradecido es de tener amigos con los que puedo debatir y conversar sobre diversos temas de todo ámbito. Recientemente he tenido el placer de tener varias conversaciones con distintas personas relacionadas con la moral y con la existencia de Dios.

    Si bien es cierto que existen una infinidad de libros y de tesis de teología al respecto, este texto es más bien para poder realizar una reflexión interna desde un gran desconocimiento, habiéndome leído únicamente y no en profundidad algunas obras dedicadas al tema de Tomás de Aquino, Dawkins, Epicuro de Samos y Leibniz. Todos ellos textos muy recomendados, pero quedándome una infinidad de libros y tesis al respecto que no he leído y que casi con total seguridad, habrán divagado ya sobre reflexiones similares a las que en este escrito expondré. 
    
    En caso de que el lector de este escrito tras leerlo conozca de alguna obra, ya sea escrita o audiovisual, que crea que pueda tener respuestas sobre algunas de las preguntas que formularé, o que pueda refutar algunas de las tesis que se expondrán, estaré más que encantado de poder investigar más al respecto para formar mejor mi opinión sobre el tema.

    Desde el comienzo quiero dejar clara mi posición, a pesar de todas las deducciones lógicas que haga sobre la bondad o maldad de Dios, mantengo mi postura agnóstica al respecto de su existencia. A día de hoy no podemos asegurar que la existencia no se trate de una perpetua sucesión de eventos, en los cuales conocidos todos los parámetros físicos del origen del universo pudiéramos determinar con infinita exactitud todos los instantes físicos pasados, presentes y futuros del mismo; en caso de este inicio existir. Pudiera darse el caso de que el determinismo defendido por grandes mentes como Brian Greene, Pierre-Simon Laplace, Immanuel Kant (con su imperativo categórico, el cual aún se me resiste como concepto que pueda entender bien y con su matiz del mundo nouménico), entre muchos otros autores destacados en física y filosofía.

    Aunque el debate sobre el determinismo, ya sea bajo una óptica compatibilista o directamente defendiendo un determinismo duro, es extremadamente interesante y un recurrente en muchas de mis conversaciones, no es el objeto de este escrito. Por tanto, a partir de este punto, supongamos como premisas iniciales que el determinismo es incorrecto, que el libre albedrío existe y que Dios existe, es decir, el siguiente escrito asumirá un acto de pura fe en algo que no es demostrable a priori con el conocimiento actual que tenemos. Como bien he comentado, no es que yo defienda ni mucho menos estas posturas, sino que soy agnóstico al respecto de ellas. Si para ti estas premisas son falsas el resto del escrito no tendrá sentido, más allá de la dialéctica lógica de las proposiciones, partiendo de una premisa errónea a modo de juego mental, y si decides optar por la vía del epojé o suspensión de juicio como es mi caso, simplemente se tratará de uno de los posibles caminos lógicos que poder explorar.

    Permíteme expresar este párrafo anterior con un simple diagrama:
    
    \begin{figure}[H]
        \centering
        \includegraphics[scale=0.5]{Imagenes/leerEscrito.png}
        \label{fig:leerEscrito}
    \end{figure}

    Partiendo de las anteriores suposiciones, Dios existe, pero, es acaso este Dios un Dios bueno, ¿o se trata de un Dios malvado que se congratula del sufrimiento que en el mundo experimentamos? El cristianismo así como otras religiones monoteístas, defienden la posición de que Dios es bueno, y algunas de ellas, no solo que sea bueno, sino que es inequívocamente identificable con el amor.

    Ante la pregunta de si Dios es bueno, uno podría responder que no, ya que este, pudiendo no permitir el mal, lo permite. El sufrimiento es un hecho irrefutable de la experiencia humana. Y no solo el causado por otro humano, si Dios es capaz de interactuar con la naturaleza, ¿por qué no impide las enfermedades causadas a humanos inocentes o bondadosos?

    Ante esta tesitura, grandes pensadores ya establecieron una forma de hablar del mal en categorías, siendo la primera el mal moral y la segunda el mal físico. Donde el mal moral es el daño o sufrimiento causado deliberadamente por seres humanos a través de acciones intencionales, como el asesinato, el robo o la crueldad, que resultan de decisiones conscientes y violan principios éticos o morales establecidas bajo la ley de Dios. Por otro lado, el mal físico se refiere al sufrimiento, daño o dolor que resulta de causas naturales o circunstancias fuera del control humano, como enfermedades, desastres naturales o accidentes.

    El mal moral, bajo la teología cristiana es "fácilmente" salvable desde un prisma lógico. Si damos como premisa válida que el mal moral existe, habría que preguntarse si Dios puede y/o quiere evitarlo. En última instancia estaríamos hablando de algo muy similar a la paradoja de Epicuro acerca de la bondad de Dios.
    
    Ante esto cabe dar una posible solución, que en principio podría ser satisfactoria para los teistas. Los razonamientos lógicos que seguiremos serán a través de la yuxtaposición de proposiciones lógicas. Supongamos que Dios existe, 

    - ¿Dios es bueno?

    - Si Dios es bueno, entonces valoraría el amor por encima de todo. Esto significaría que desearía que nos amáramos entre nosotros y que lo amáramos a Él.

    - ¿Es aceptable que un amor sea forzado mediante coacción o extorsión si partimos de la idea de que Dios es bueno?

    - Un amor forzado o coaccionado no puede considerarse verdadero amor. En el mejor de los casos, sería improbable o surgiría a partir de un trastorno, como el síndrome de Estocolmo.

    - ¿Dios tiene la capacidad de quitarnos la libertad y forzarnos a amarlo a Él y a los demás?

    - Aunque Dios tiene el poder de privarnos de libertad, eligió no hacerlo. Si hubiera obligado a la humanidad a amarlo, sería un Dios malvado. En cambio, permitió que fuéramos libres, porque desea que lo amemos voluntariamente, sin imposición.

    - La bondad de Dios se refleja en su deseo de que seamos libres, para que podamos amarlo sin coacción.

    - Sin embargo, el mal existe, a menudo provocado por otros seres humanos. ¿Es Dios malvado por no eliminar ese mal?

    - Si Dios eliminara el mal moral, esto implicaría que también eliminaría la libertad de aquellos que lo causan. Por tanto, si Dios nos privara de libertad, sería un Dios malvado.

    - En conclusión, aunque Dios permite la existencia del mal moral, no es malvado, porque hacerlo preserva nuestra libertad.

    Esta yuxtaposición de proposiciones lógicas es abordable para debates separados para cada una de ellas, ya que buscando bien argumentos lógicos, para todas se puede encontrar contra argumentaciones ciertamente interesantes. No obstante, personalmente (aunque entiendo que quien lea esto pueda no estar de acuerdo) me parecen razonamientos lógicos muy sólidos.

    Para el mal físico no obstante, la cosa cambia, y todo se pone mucho más interesante de explorar. Una posible respuesta que me he encontrado para salvaguardar la bondad de Dios con respecto al mal físico (y a veces también para el moral), es que aunque Dios cause sufrimiento a una persona, desde el prisma cristiano, la muerte no se trata del final, sino que solo es parte del camino. Es decir, aunque tu vida haya estado llena de sufrimiento constante y continuo, tras la muerte, si así lo deseas estará Dios esperándote para conducirte hacia la vida eterna de amor.

    Para esta respuesta mi contra argumentación más usada es la siguiente. Imaginemos que la vida del ser humano es eterna, siendo inmortales, y que un humano, al que llamaremos por ejemplo Bob, tiene la capacidad de hacer y deshacer. Durante un periodo de tiempo finito encierra en jaulas minúsculas a parte de la humanidad, y estas jaulas tienen ventanas al exterior donde pueden ver al resto de la misma, y como su sufrimiento no es universal, sino localizado. Los humanos encerrados sufren constantemente y toda su vida es una agonía, tanto por comparación como por experimentación. A algunos les han dicho que Bob les tiene encerrados porque tiene un plan para que luego puedan vivir sin sufrimiento, y por tanto, aunque viven en sufrimiento y agonía mantienen la experanza de promesas futuras. Pasado un tiempo finito, Bob les libera y les dice que ahora pueden vivir tranquilos felices y disfrutando de la existencia eterna en un reino en el que Bob es Rey y señor de todo. 
    
    Ante este hecho, una de las humanas encerradas, a la que llamaremos Alice, le pregunta a Bob el porque de su sufrimiento durante tanto tiempo, si podía haberlo evitado. Bob es el responsable de su sufrimiento y el causante de todo el mal que Alice ha padecido durante todos estos años pudiendo evitarlo, y sin embargo Bob le está pidiendo que le perdone por haberla tenido encerrada y que acepte que sea su futuro Rey y señor para el resto de la eternidad. Esto a todas luces es una injusticia, Bob se trata de un criminal por cometer atrocidades contra parte de la humanidad, debería ser juzgado por tales crímenes.

    Incluso aunque el fin de Bob en este caso sea un fin noble y bueno, la vida eterna en amor y armonía, los medios para realizarlos parecen injustificados por el mal causado. El fin de Hitler era crear un mundo en el que todos los humanos pudieran ser felices y prósperos, pero los medios que quería utilizar pasando por el hecho de que a muchos humanos no los considerara humanos y que quisiera deshacerse de todos los que estaban en contra de su visión del mundo perfecto, eran completamente injustificados. Un Dios que defienda que el fin justifica cualquier medio por atroz que este sea, debería ser un Dios malvado.

    Dado esto, las conclusiones que podríamos sacar son las siguientes: 
    \begin{itemize}
        \item[-] Dios es malvado, un criminal que justifica cualquier medio para llegar a un fin.
        \item[-] Dios creo el mal físico sin querer y no tiene poder para eliminarlo.
        \item[-] Dios no existe.
    \end{itemize} 

    No obstante quizá exista una vía extra que permita salvaguardar la bondad de Dios y que a nivel lógico, pueda tener un sentido amplio. Tanto Leibniz como San Agustín, exploraron la idea de que vivimos en el mejor de los mundos que se podían haber creado. Bajo esta premisa cabe preguntarse cómo es posible que el mejor mundo posible es uno en el que se permita un mal físico tan evidente como el que tenemos en el mundo en el que nos encontramos ahora. 
    
    Ante esta pregunta, San Agustín respondió que, en última instancia, contribuye a un bien mayor dentro del plan divino. Es decir, desde nuestra visión humana no somos capaces de entender como es posible que ese mal genere un bien mayor, pero si tuviéramos una visión cosmológica amplia de toda la historia pasada, presente y futura del universo y la eternidad, veríamos que es el mejor universo posible. 
    
    Quizá sea el único universo en el que la humanidad acepta tras su muerte a Dios y por tanto acepta la felicidad y el amor eterno (si esto existiese). Quizá es el único universo en el que no nos extinguimos bajo guerras violentas, enfermedades o desastres naturales. Quizá es el único universo del cual somos capaces de transcender nuestra humanidad y llegar a un plano de la existencia diferente al que ahora conocemos. Sea cual sea el bien mayor del que se habla, no podremos saber si este es el único universo capaz de alcanzarlo hasta que lo alcancemos si es que lo hacemos, y en ese punto, puede que estemos agradecidos como humanidad de que este sea el universo que se creo.

    No obstante, eso sigue siendo justificar cualquier medio para llegar a un fin. Vale, es un fin infinitamente bueno, pero es un fin al que se ha llegado por unos medios evitables y que no se han evitado. Por tanto, ¿es Dios un criminal malvado?

    Aquí es donde expongo mi tesis al respecto. Vamos a suponer en la analogía anterior, que Alice no es que estuviera en una jaula, sino en una vida miserable. Al morir tras tanto sufrimiento se encuentra con Dios. Dios le explica que todo lo que ha pasado era parte de su plan, le muestra el fin último, le explica lo mucho que la ama y lo mucho que lamenta que haya tenído que pasar por eso, pero que fue la única forma de alcanzar el fin último que únicamente Dios conocía. Ante esa verdad revelada Alice podrá decidir que está conforme con la explicación, que lo entiende, que ella habría hecho lo mismo, que le agradece haber preparado ese lugar de amor infinito y plenitud que es el cielo y que le permita entrar. O, podrá decir, vale, este cielo al que me invitas es estupendo, pero eres tu el culpable de todo mi sufrimiento aunque me dices que me querías, ¿quién va a asumir esa responsabilidad?

    Ante esto, y dado que me gustaría pensar que en caso de existir un Dios, este es bueno, mi deducción lógica es que lo que ocurriría sería lo siguiente. Dios, le diría que en última instancia la ama, en el sentido amplio de la palabra, que no hay nada que desease más que ver como se une en el cielo al resto de la humanidad y que asumiría cualquier castigo que pudiera imponerle, cualquiera que fuera, por amor lo aceptaría para que ella pudiera ser feliz. Alice en ese momento podría desearle el mayor sufrimiento que pudiera imaginarse, el mayor dolor que pudiera imaginarse y que muriera entre terribles sufrimientos, ante lo que Dios, en un infinito amor hacia Alice respondería, ''acepto el castigo''. 
    
    Este, bajo mi punto de vista, es uno de los puntos fuertes del cristianismo a un nivel teológico (al menos las ramas que aceptan la trinidad monoteísta), ya que su Dios literalmente acepta morir a manos de la humanidad. Dios, tras ese momento decide bajar a un punto de la historia de la humanidad en donde por predicar amor a gente a la que ama en el sentido amplio, le condenan a muerte, le torturan y muere en una infinita agonía. Además, después de todo, vuelve de nuevo tras su muerte para afirmar que les perdona por todo lo que han hecho y que no se queden con la duda, y no solo a ellos, sino a todos los que vienen en el futuro. 
    
    En la analogía, Alice experimentaría la vida de Dios encarnado en Jesús como si se tratase de una película, pero vivida en sentido amplio desde una perspectiva superior. 
    
    Bajo esta posibilidad, en el momento en el que mueres y le exiges responsabilidades a Dios, por cada persona que le desea el peor de los males, por amor decide vivir una y otra vez esa agonía, ese dolor, y ese perdón todas y cada una de las veces. Ante eso, sospecho que Alice podría perdonar y entender que Dios efectivamente es amor, y aun cometiendo actos malvados los cometió por no tener más remedio, bajo unas premisas que ahora no somos capaces de entender.

    Esta es solo una interpretación posible al teísmo para salvaguardar la bondad de un hipotético Dios existente y bajo un prima cristiano. Mi punto de vista al respecto, como expuse al principio, es que podría ser así, o podría no ser así, y que no podremos saber la autentica respuesta a esto hasta morir. Al menos de momento, ya que tampoco sé si dentro de miles de años la ciencia, la filosofía u otras disciplinas que aún nos son desconocidas, podrán afirmar con total certeza (aunque lo dudo) que Dios no existe, o que es malvado, o que es bondadoso pero de otra forma. Pero esta interpretación, al menos a mi, me sirve para poder sortear la paradoja de Epicuro usando premisas lógicas plausibles y razonables para llegar a entendimientos con personas teístas conocidas.

    Como siempre, agradecer que hayas leído mis reflexiones y conclusiones, y agradecer a todas las personas con las que he podido debatir y charlar sobre estos temas. Para finalizar cómo con el resto de mis escritos,
    
    Y tú, ¿qué piensas?

    \includegraphics{Imagenes/FirmaMartin.png}
    
    
    
    
   
    
    
    